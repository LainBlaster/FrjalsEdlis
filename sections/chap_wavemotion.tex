% Chap wave motion
\chapter{Bylgjuhreyfing}
Bylgjuhreyfing er þegar hlutur hreyfist með reglulegri tíðni, t.d.
sveiflur á pendúl og útbreiðsla á bylgju í plani. Þá er hægt að nota
svipaðar jöfnur og gert er í kaflanum um hringhreyfingu. Föll sínus
og kósínus henta mjög vel til þess. Samt er nauðsyn að koma með nýja
fasta og stærðir sem lýsa hegðun bylgjunnar.

Þá er einhver bylgja með hámarks útslag frá miðstöðu ($\uwaveampA$) 
og hornhraða ($\uwaveangularomega$). Þá er
klassískri bylgju oft lýst með fallinu
\[
	\ux(\utime) = \uwaveampA \cos (\uwaveangularomega \utime)
\]
þetta er ein leið til að lýsa slíku falli, það er til nokkrar útgáfur
sem er notaðar eftir ýmsum aðstæðum. Það eru nokkur hugtök sem bylgjum
er ljáð í téð
\begin{description}
	\item[Ölduhæð, sveifluvídd] Sem er hámarkið sem bylgjan nær frá miðju sinni. 
		Sem er mesta útslag sem hlutur í hreyfingu getur leyft sér frá punkti
		sem oft nefndur miðstaða.
	\item[Bylgjuhraði] Sá hraði sem bylgja breiðir út sér með
	\item[Bylgjulengd] Lengin sem bylgja nær að breiða úr sér í einni sveiflu
	\item[Hornhraði, horntíðni] Er fasti sem gefur upplýsingar um hversu marga
		radíana í ferli bylgjunnar eru farnir á hverri sekúndu.
	\item[Tíðni, sveiflutíðni] Er fasti sem gefur upplýsingar umgangar
		í ferli bylgjunnar eru farnir á hverri sekúndu. Svipað og fara heilan
		hring í hringferlinum.
\end{description}
\todo[color=blue!25!white]{Avoid this list, seemes to defeat its own purpose }%

\begin{figure}
	\begin{tikzpicture}[domain=0:4]
	  \draw[very thin,color=gray] (-0.1,-1.1) grid (3.9,3.9);
	  \draw[->] (-0.2,0) -- (4.2,0) node[right] {$x$};
	  \draw[->] (0,-1.2) -- (0,4.2) node[above] {$y$};
	  \draw plot[id=sin] function{sin(x)} node[right] {$f(x) = \sin x$};
	\end{tikzpicture}
	\caption{Mynd af helstu stærðum sem eru tengd bylgjuhreyfingu}
\end{figure}
\todo[color=blue!25!white]{Finish current picture without using pgf functions}%

\section{Sveiflur gorma}
Þegar gormur sveiflast umhverfis miðstöðu, þá togar hann hlutinn aftur í átt
að miðstöðu. Þetta þýðir í kraftasamhengi að
\begin{equation}
	\uforce_\text{gormur} = \uspringconstk \ulengthx
\end{equation}
og ber að nefna stefna gormkraftsins er alltaf í átt að miðstöðu, þetta er einungis
stærð gormkrafts. Þá er hægt að
leika sér smáveigis með slíkar jöfnu, það er hægt að umskrifa til
\[
	\umass \uaccelea = -\uspringconstk \ulengthx
\]
neikvætt formerki hérna táknar að krafturinn beinist að miðstöðu. Við vitum
að hröðun er breyting á hraða yfir tíma, og hraði er breyting á staðsetningu yfir
tíma. Sem gefur
\begin{align*}
	\umass \uaccelea(\utime) &= -\uspringconstk \ulengthx(\utime) \\
	\umass \frac{\ud \uspeed(\utime)}{\ud \utime} &= -\uspringconstk \ulengthx(\utime) \\
	\umass \frac{\ud^2 \ulengthx(\utime)}{\ud \utime^2} &= -\uspringconstk \ulengthx(\utime)
\end{align*}
sem er hægt að umskrifa til
\[
	\frac{\ud^2 \ulengthx(\utime)}{\ud \utime^2} = -\frac{\uspringconstk}{\umass} \ulengthx(\utime)
\]
þetta er diffurjafna sem hefur sett af lausnum sem eru samsetning af sínus eða
kósínus lausnum. Hins vegar þá var í byrjun kaflans þá kom ágiskun á hvað gæti
verið heppileg lýsing á bylgju sem er
\[
	\ux(\utime) = \uwaveampA \cos (\uwaveangularomega \utime)
\]
þá er innsetning í diffurjöfnu
\begin{align*}
	\frac{\ud^2 \ulengthx(\utime)}{\ud \utime^2}
		&= \frac{\ud}{\ud \utime}\left( \frac{\ud \ulengthx(\utime))}{\ud \utime} \right) \\
		&= \frac{\ud}{\ud \utime}\left( -\uwaveampA \uwaveangularomega \sin{(\uwaveangularomega \utime)}  \right) \\
		&= -\uwaveampA \uwaveangularomega^2 \cos{(\uwaveangularomega \utime)} \\
		&= - \uwaveangularomega^2 \ux(\utime)
\end{align*}
og þá er diffurjafnan líka
\begin{align*}
	-\frac{\uspringconstk}{\umass} \ulengthx &= - \uwaveangularomega^2 \ux(\utime) \\
	\frac{\uspringconstk}{\umass} &= \uwaveangularomega^2 \\
	\uwaveangularomega^2 &= \frac{\uspringconstk}{\umass} \\
	\uwaveangularomega &= \sqrt{\frac{\uspringconstk}{\umass}}
\end{align*}
sem er áhugaverð niðurstaða, núna er hægt að finna hornatíðni hlutars á enda
gorms með einungis upplýsingar um massa og kraftstuðull gorms
\begin{align}
	\uwaveangularomega &= \sqrt{\frac{\uspringconstk}{\umass}}
\end{align}
sem er líka hægt að umskrifa til að vera sveiflutími
\begin{align}
	\uorbitaltime &= 2 \pi \sqrt{\frac{\umass}{\uspringconstk}}
\end{align}
þá eru komnar sett af breytum sem geta lýst furðulega miklu varðandi innihaldi
sveifluhreyfingar, án þess að taka sérstakt tilfelli af gormi fyrir.

\begin{formalexample}
Gormur liggur í láréttu og er fest við massa sem er $5 \ukilo\ugramm$, kraftstuðull
gormsins er $140 \uspringconstnm$, hver er horntíðni, sveiflutíðni og sveiflutími
gormsins?
\\[4 ex]
Þar sem gefið er bæði kraftstuðull og massir þá er horntíðni
\begin{align*}
	 \uwaveangularomega &= \sqrt{\frac{\uspringconstk}{\umass}}
		= \sqrt{\frac{140 \uspringconstnm}{5 \ukilo\ugramm}}
		= 5,29 \uhertzs
\end{align*}
síðan er hægt að nota til að finna tíðni með
\begin{align*}
	\ufrequency &= \frac{\uwaveangularomega}{2 \pi}
		= \frac{5,29 \uhertzs}{2 \pi}
		= 0,842 \uhertz
\end{align*}
og sveiflutíma með
\begin{align*}
	\uorbitaltime &= \frac{1}{\ufrequency}
		= \frac{1}{0,842 \uhertz}
		= 1,19 \usec
\end{align*}
\end{formalexample}
Sem stærð er hægt að nýta horntíðni á margvíslegan máta, t.d. er sveiflutíðni
($\uwaveangularomega$) stærð sem getur lýst hegðun efna. Þá eru atóm með bæði
fráhrindi- og aðdráttarkrafta, sem er oft lýst með eiginleikum gormsins sem
líkan af hegðun þeirra. Þá er titringur af atómum með tíðni á stærðargráðunni
$10^{13} \uhertz$. Sem myndi gefa fyrir massa koparatóms
\begin{align*}
	2 \pi \ufrequency &= \sqrt{\frac{\uspringconstk}{ \umass}} \\
	2 \pi \cdot 10^{13} \uhertz &= \sqrt{\frac{\uspringconstk}{ \uEE{1.05}{-25} \ukilo\ugramm }} \\
	\uspringconstk &= \left(2 \pi \cdot 10^{13} \uhertz \right) \cdot \uEE{1.05}{-25} \ukilo\ugramm \\
		&= 41600 \uspringconstnm
\end{align*}
sem er há tala, en í raun sýnir af hverju málmar eru sterkir og hafa góða leiðni
fyrir hita. Líka er hraði hljóðs margfalt hærri í málmum en í lofti, nákvæmlega vegna
stífleika ,,gormana''.


\subsection{Orka sveifluhreyfingar}
Þar sem við fengum heppilega lýsingu á því hvernig bylgja hagar sér við góða
ágiskun í fyrri köflum, þá er hægt nota hana til að sýna að orkan sem sveiflan
hefur er föst, óbreytt. Fyrst þarf að finna stöðuorku gormsins, almennt gildir að
vinnan framkvæmd sem verður að stöðuorku er
\[
	\upotentialu = \int_{\ulengthx_1}^{\ulengthx_2} \uforce \ud \ulengthx
\]
eða með krafti gorms verður það
\begin{align*}
	\upotentialu &= \int_{0}^{\ulengthx_2} \uspringconstk \ulengthx \ud \ulengthx
		= \left[ \frac{1}{2} \uspringconstk \ulengthx^2 \right]_{0}^{\ulengthx_2}
		= \frac{1}{2} \uspringconstk \ulengthx_2^2
		= \frac{1}{2} \uspringconstk \ulengthx^2
\end{align*}
en áður er áfram haldið, þá er góð hugmynd að hafa fundið hraðan sem massinn
á enda gormsins hefur
\begin{align*}
	\uspeed(\utime) &=  \frac{\ud \ulengthx(\utime)}{\ud \utime}
	= -\uwaveampA \uwaveangularomega \sin{(\uwaveangularomega \utime)}
\end{align*}
og vélræn orka gormsins er
\begin{align*}
	\umechenergy &= \ukinetick + \upotentialu \\
		&= \frac{1}{2} \umass \uspeed^2 + \frac{1}{2} \uspringconstk \ulengthx(\utime)^2 \\
		&= \frac{1}{2} \umass \left(-\uwaveampA \uwaveangularomega \sin{(\uwaveangularomega \utime)} \right)^2
			+ \frac{1}{2} \uspringconstk \left( \uwaveampA \cos{(\uwaveangularomega \utime)} \right)^2 \\
		&= \frac{1}{2} \umass \uwaveampA^2 \uwaveangularomega^2 \sin^2{(\uwaveangularomega \utime)}
			+ \frac{1}{2} \uspringconstk \uwaveampA^2 \cos^2{(\uwaveangularomega \utime)}
\end{align*}
þá er hægt að nýta $\uwaveangularomega^2 = \frac{\uspringconstk}{\umass}$ til að
\begin{align*}
	\umechenergy &= \ukinetick + \upotentialu \\
		&= \frac{1}{2} \umass \uwaveampA^2 \uwaveangularomega^2 \sin^2{(\uwaveangularomega \utime)}
			+ \frac{1}{2} \umass \uwaveangularomega^2 \uwaveampA^2 \cos^2{(\uwaveangularomega \utime)} \\
		&= \frac{1}{2} \umass \uwaveampA^2 \uwaveangularomega^2 
			\left( \sin^2{(\uwaveangularomega \utime)} + \cos^2{(\uwaveangularomega \utime)} \right) \\
		&= \frac{1}{2} \umass \uwaveampA^2 \uwaveangularomega^2 
			\left( 1 \right) \\
		&= \frac{1}{2} \umass \uwaveampA^2 \uwaveangularomega^2
\end{align*}
Sem þýðir að heildarorka gorms er ávalt föst stærð, óháð því hvar á ferli sveiflunar
gormurinn er, sem gefur
\begin{equation} \label{eqn:wavemotion:springenergy}
	\umechenergy = \frac{1}{2} \umass \uwaveampA^2 \uwaveangularomega^2
		= \frac{1}{2} \uspringconstk \uwaveampA^2
\end{equation}

\begin{formalexample}
Gormur hefur kraftstuðullinn $160 \uspringconstnm$ og sveifluvíddin er $0,2 \umeter$. Hver
er heildarorka gormsins?
\\[4 ex]
Þá er orkan
\[
	\umechenergy = \frac{1}{2} \uspringconstk \uwaveampA^2
		= \frac{1}{2} \cdot 160 \uspringconstnm \cdot \left( 0,2 \right)^2
		= 3,2 \ujoule
\]
\end{formalexample}
\begin{formalexample}
Gormur með massann $2 \ukilo\ugramm$ er teygður $0,1 \umeter$ frá miðstöðu og gefinn 
hraðinn $1,2 \uspeedms$ frá miðstöðu.
Kraftstuðull gormsins er $160 \uspringconstnm$. Hver er sveifluvídd gormsins?
\\[4 ex]
Heildarorkan er
\begin{align*}
	\umechenergy &= \ukinetick + \upotentialu \\
		&= \frac{1}{2} \umass \uspeed^2 + \frac{1}{2} \uspringconstk \ulengthx(\utime)^2 \\
		&= \frac{1}{2} \cdot 2 \ukilo\ugramm \cdot \left( 1,2 \uspeedms \right)^2 
			+ \frac{1}{2} \cdot 160 \uspringconstnm \cdot \left( 0,1 \umeter \right)^2 \\
		&= 1,44 \ujoule 
			+ 0,8 \ujoule \\
		&= 2,24 \ujoule
\end{align*}
þar sem heildarorkan er alltaf sú sama, þá er hægt að setja orkuna sem er
reiknuð á einum stað í sveifluhreyfingunni til að vera jöfn heildarorkunni
fundin í jöfnu \ref{eqn:wavemotion:springenergy}
\begin{align*}
	2,24 \ujoule &= \frac{1}{2} \uspringconstk \uwaveampA^2 \\
	\frac{2,24 \ujoule}{\frac{1}{2} \uspringconstk } &= \uwaveampA^2 \\
	\uwaveampA^2 &= \frac{2,24 \ujoule}{\frac{1}{2} \cdot 160 \uspringconstnm }\\
	\uwaveampA^2 &= 0.028 \umeter^2\\
	\uwaveampA &= \sqrt{0.028 \umeter^2} \\
		&= 0.167 \umeter = 16,7 \ucenti\umeter
\end{align*}
Sem er sveifluvíddin.
\end{formalexample}

\section{Pendúll}
Það er hægt að umskrifa þann kraft sem pendúll verður fyrir vegna þyngdarkrafts til að
vera með sömu hegðun og gormur. Hins vegar er það einungis hægt fyrir sveiflur sem eru
ekki sérlega stórar. %
\todo[color=blue!25!white]{Add a picture showing the approximation}%
Þetta er nálgun á hegðun en virkar furðulega vel. Þá er krafturinn 
sem verkar á massa í spotta venjulega
\begin{align*}
	\uforce_\text{pendúll} &= \uforce_\uacceleg \sin{\theta}
\end{align*}
hins vegar með nálgun verður
\begin{align*}
	\uforce_\text{pendúll} &= \uforce_\uacceleg \frac{\ulengthx}{\ulengthl}
\end{align*}
og á sama máta og með gorminn verður
\begin{align*}
	\umass \uaccelea 
		&= -  \frac{\uforce_\uacceleg}{\ulengthl} \ulengthx \\
	\umass \uaccelea 
		&= -  \frac{\umass \uacceleg}{\ulengthl} \ulengthx \\
	\uaccelea 
		&= - \frac{\uacceleg}{\ulengthl} \ulengthx \\
	\frac{\ud^2 \ulengthx}{\ud \utime^2}
		&= - \frac{\uacceleg}{\ulengthl} \ulengthx
\end{align*}
sem þýðir að fyrir pendúl að
\begin{align}
	\uwaveangularomega &= \sqrt{ \frac{\uacceleg}{\ulengthl} }
\end{align}
og gefur að sveiflutíminn er
\begin{align}
	\uorbitaltime &= 2 \pi \sqrt{ \frac{\ulengthl}{\uacceleg} }
\end{align}
\begin{formalexample}
Kúla hengur í $2 \umeter$ löngu bandi, hver er horntíðni og sveiflutími
kúlunnar?
\\[4 ex]
Þá er horntíðni
\begin{align*}
	\uwaveangularomega &= \sqrt{ \frac{9,8 \uaccelems}{2 \umeter} }
		= 2,21 \uhertzs
\end{align*}
og sveiflutíman
\begin{align*}
	\uorbitaltime &= 2 \pi \sqrt{ \frac{2 \umeter}{9,8 \uaccelems} }
		= 2,84 \usec
\end{align*}
\end{formalexample}

\section{Einfaldar bylgjur}
Bylgjur hafa getuna til að breiða úr sér með ákveðnum hraða, þá er sagt að
hún hafi útbreiðsluhraða. Sem hluti af því þarf að skilgreina bylgjulengd $\uwavelength$ og
sveiflutíma $\uorbitaltime$. Bylgjulengd er veglengdin sem bylgjan ferðast til að fara frá einum
bylgjutoppi til annars~\footnote{Það er réttar að segja það er vegalengdin
sem bylgjan ferðast á einni sveiflu sem er $2 \pi$ í radíönum} og sveiflutíminn
er tíminn sem tekur að fara frá byrjunarstöðu til útslags báðum megin við miðstöðu
og aftur til byrjunarstöðu.
Bylgjuhraði er skilgreindur til að vera
\begin{equation}
	\uwavevelocity = \frac{\uwavelength}{\uorbitaltime} = \uwavelength \ufrequency
\end{equation}

\todo[color=blue!25!white]{Add a picture for the wave, adding wavelength and
period symbols}%

Einföldum bylgjum er oft lýst á tvö máta, sem langbylgjur og sem þverbylgjur.
Langbylgjur hafa bylgjueiginleika \emph{samsíða} útbreiðslustefnu á meðan
þverbylgjur hafa bylgjueiginleika \emph{þvert} á útbreiðslustefnu. Sem dæmi eru
vatnsbylgjur þverbylgjur og hljóðbylgjur eru langbylgjur.
\todo[color=blue!25!white]{Add a picture for the longditudional and transversive
waves}%

\subsection{Samliðun bylgja}
Bylgjur hafa þann eiginleika að geta tvinnast saman, sem leiðir af sér aukningu
eða minnkun af útslagi bylgjunnar. Gagnleg samliðun er þegar tvær (eða fleiri)
bylgjur auka útslagið og eyðandi samliðun er þegar tvær (eða fleiri) ná að
draga úr útslagi hvors annars. Dæmi um slíkt er gefið of í myndrænu formi en
í jöfnuformi þá er það
\begin{align*}
	\ux_\text{AB}(\utime) &= \ux_\text{A}(\utime) + \ux_\text{B}(\utime) \\
		&= \uwaveampA_\text{A} \cos (\uwaveangularomega \utime + \uphase_\text{A}) +
			\uwaveampA_\text{B} \cos (\uwaveangularomega \utime + \uphase_\text{B}) \\
\end{align*}
sem er í það minnsta lagi ekki álítanlegt form, hérna er $\uphase_\text{A}$ og
$\uphase_\text{B}$ fasahliðrun fyrir sitthvora bylgjuna. Í meira daglegu tali
þá er fasahliðrun svipað og breyta tímasetningunni sem bylgjan leggur afstað.
Til að búa til eyðandi samliðun þarf ákveðið skilyrði að koma upp, bylgjurnar
þurfa að vera úr fasa (s.s. byrja ekki á sama tíma) og með jafnstórt hámarks útslag.
Til að sjá slíkt er hægt að hafa
$\uphase_\text{AB} = \uphase_\text{A} - \uphase_\text{B} = 180 \udeg$ og
$\uwaveampA_\text{A} = \uwaveampA_\text{B}$ sem gefur
\begin{align*}
	\ux_\text{AB}(\utime) &= \ux_\text{A}(\utime) + \ux_\text{B}(\utime) \\
		&= \uwaveampA_\text{A} \cos (\uwaveangularomega \utime) +
			\uwaveampA_\text{A} \cos (\uwaveangularomega \utime + 180 \udeg) \\
		&= \uwaveampA_\text{A} \cos (\uwaveangularomega \utime) -
			\uwaveampA_\text{A} \cos (\uwaveangularomega \utime) \\
		&= 0 \\
\end{align*}
sem gefur að bylgjur sem er úr fasa með $180 \udeg$ og með sömu sveifluvídd hafa
lengur ekki áhrif á nokkuð. Sem þykir dálítið sérstakt en í raun er þetta kunnulegt
fyrirbrigði úr daglegu lífi þar sem við getum stundum heyrt hljóð ,,hverfa''.

\todo[color=blue!25!white]{Add picture for interference, showing constructive
and destructive interference}%

\section{Doppler áhrif}
Doppler áhrif koma þegar bylgjulengdin breytist vegna þess að sendandi bylgjunar
er á hreyfingu. Þá er hægt að finna hver nýja bylgjulengdin verður sem hlustandi
heyrir. Það þarf einmitt að gera greinamun á því hver sendir og móttekur
bylgjuna. Daglega er þetta kunnugt þegar bílar keyra frammhjá manni, t.d.
sírenur hljóma ,,dýpri'' þegar sjúkrabíll keyrir frá manni en ,,skrækari''
þegar sjúkrabíll keyrir að manni. Þá þarf að koma með skilgreiningar 
á því hver er sendandi og hver er móttakandi.
\begin{align*}
	\uwavevelocity &= \uwavelength_{\text{send}} \ufrequency_{\text{send}} \\ 
	\uwavevelocity &= \uwavelength_{\text{mót}} \ufrequency_{\text{mót}}
\end{align*}
hraði sendanda hefur áhrif á útbreiðslu með
\begin{align*}
	\uwavelength_{\text{send}} &= 
		(\uwavevelocity \pm \uspeed_{\text{send}} ) \uorbitaltime_{\text{mót}}
\end{align*}
þegar sendandi stefnir að móttakanda þá dregst hraðinn frá og þegar sendandi
stefnir frá móttakanda bætist hraðinn við. Sem gefur að tíðni sendana er
\begin{align*}
	\ufrequency_\text{send} &= \frac{ \uwavevelocity }{
		 (\uwavevelocity \pm \uspeed_\text{send} ) \uorbitaltime_\text{mót} } \\
	&= \frac{ \uwavevelocity} {
		 \uwavevelocity \pm \uspeed_\text{send} } \ufrequency_\text{mót}
\end{align*}
þetta er sértilfelli þegar annar aðilinn er kyrr og hinn er á hreyfingu, þá
er $\uwavevelocity$ hraði bylgjunar í miðli%
~\footnote{Miðill getur t.d. verið loft sem hljóðið berst með eða sjálft
tómarúmið sem ljósið berst með.}
\todo[color=blue!25!white]{Add a picture for a observer that is static while the
transmitter is in motion, showing the change in wavelength}
Sem gefur að tíðnin sem sendandin hefur er
\begin{align}
	\ufrequency_\text{send} &= \frac{ \uwavevelocity } {
		 \uwavevelocity \pm \uspeed_\text{send} } \ufrequency_\text{mót}
\end{align}
hins vegar er hægt að láta móttakanda vera líka á ferð, þá verður uppsetningin svipuð
nema að bylgjulengd móttakanda breytist líka eins og hjá sendanda.
\begin{align*}
	\uwavelength_{\text{mót}} &= 
		(\uwavevelocity \pm \uspeed_\text{mót} ) \uorbitaltime_{\text{mót}}
\end{align*}
