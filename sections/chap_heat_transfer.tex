% Chap angular motion
\chapter{Varmarýmd}
Þegar hitamismunur á sér stað þá mun heitari hluturinn reyna lækka hitastigið
sitt og kaldari hluturinn mun reyna hækka hitastigið sitt. Það sem gerist er
að orka færist á milli hlutana og sú orka hefur nafnið varmi.

Varmi er orka, en er nánar tiltekið orka sem tengist hitastigi hluta. Því er
varmaflæði líka orkuflæði, hlutir geta gefið frá sér varma og tekið til
sín varma. S.s. það er geta sem hlutir hafa mismunandi mæli.

Varmarýmd er stærð táknar magnið af orku fyrir hvert kelvin~\footnote{Hægt 
líka að segja hluturinn rúmar orku á hverja kelvin breytingu} 
sem hlutur breytir hita sínum. Í jöfnuformi þá er varmarýmd
\begin{equation}
	\uheatcapacity = \frac{\uthermaleng}{\Delta \utemperature}
\end{equation}
síðan er hægt að mæla varmarýmd hluta með að þekkja magnið af orku sem er notað
til að hita hlutinn upp. Einingin fyrir varmarýmd er $\uheatcapacityjk$ eða
Joule per Kelvin.

\begin{formalexample}
Ílát er hitað upp með rafhitara sem hefur aflið $2 \ukilo\uwatt$, uþb. $80\%$ af
orku rafhitarns fer í að hita ílátið, eftir að rafhitarinn er búin að vera
í gangi í $2 \umin$ þá er hiti ílátsins búinn að hækka um $20 \ukelvin$.
Hver er varmarýmd ílátsins?
\\[4 ex]
tíminn sem hitarinn er í gangi eru $2 \umin = 120 \usec$ og orkan sem hann lætur
í té er
\[
	\Delta\upotentialu = \upower \Delta\utime 
		= \uEE{2}{3} \uwattjs \cdot 120 \utime
		= 240 \ukilo\ujoule
\]
og af þessari orku, þá er $\uthermaleng = 0,80 \cdot 240 \ukilo\ujoule 
= 192 \ukilo\ujoule$ sem gefur að varmarýmdin er
\begin{align*}
	\uheatcapacity &= \frac{\uthermaleng}{\Delta \utemperature} \\
		&= \frac{\uEE{192}{3} \ujoule}{20 \ukelvin} \\
		&= 9600 \uheatcapacityjk 
\end{align*}
\end{formalexample}

\section{Eðlisvarmi}
Eðlisvarmi er sama hugtak og varmarýmd nema þá er tekið mark á massa hlutsins
sem hefur varmarýmdina. S.s. eðlisvarmi er varmarýmd per kílógramm. Eðlisvarmi er
gefin til að vera
\begin{equation}
	\uspecificheat = \frac{\uthermaleng}{\umass \Delta\utemperature}
\end{equation}
þar sem einingin fyrir eðlisvarma er $\uspecificheatjkgk$. Önnur heppileg
leið til að sýna sömu jöfnu er
\[
	\uthermaleng = \umass \uspecificheat \Delta \utemperature
\]
en þessi jafna er í raun útleitt form af varmarýmd per massa.
\begin{formalexample}
Hversu mikla orku þarf til þess að hita 2 lítra af vatni um $30 \ukelvin$? 
Eðlisvarmi vatns er $4186 \uspecificheatjkgk$.
\\[4 ex]
Varminn sem vatnið tekur við er
\begin{align*}
	\uthermaleng &= \umass \uspecificheat \Delta \utemperature \\
		&= 2 \ukilo\ugramm \cdot 4186 \uspecificheatjkgk \cdot 30 \ukelvin \\
		&= 251 \ukilo\ujoule 
\end{align*}
\end{formalexample}

\section{Gagnleg not af varmarýmd}
Dæmi um nýtingu varmarýmdar er oftast til að finna breytingu í hitastigi hluta
miðað við magnið af orku sem er gefin eða þeginn sem varmi. Það sem er oft góð
hugmynd að áætla er að varmi sem hlutur getur látið frá sér fer allur í hlutinn
sem þiggur. Sem gefur
\[
	\uthermaleng_\text{gefin} = \uthermaleng_\text{þeginn}
\]
til að byrja með virðist þetta ekki vera einstaklega hagnýtt fyrirbæri. Vonin er
að sýna hvernig þetta litla samhengi leggur drög að góðu verkfæri.

\begin{formalexample}
$20 \ugramm$ af ís við bræðslumark er færður í skál með $0,5 \uliter$ af $20 
\udegcent$ heitu vatni. Hvert er lokahitastigið á vatninu með bráðnaða ísnum?
\\[4 ex]
Vatnið sem ísinn er lagður mun tapa varma sem ísinn mun allur nýta, þeas. vatnið
gefur varman sem ísinn þiggur. Bæði bráðnaði ísinn og vatnið munu enda í sama 
lokahitastigi. Orka sem krefst að breyta hitastigi vatnsins er
\begin{align*}
	\uthermaleng_\text{vatn} &= \umass_\text{vatn} \uspecificheat_\text{vatn}\Delta \utemperature_\text{vatn} \\
		&= 0,5 \ukilo\ugramm \cdot 4186 \uspecificheatjkgk \cdot \Delta \utemperature_\text{vatn} \\
		&= 2093 \uheatcapacityjk \cdot \Delta \utemperature_\text{vatn}
\end{align*}
varminn sem ísinn þiggur er
\begin{align*}
	\uthermaleng_\text{ís} &= \ulatentheat \umass_\text{ís} 
		+ \umass_\text{ís} \uspecificheat_\text{vatn}\Delta \utemperature_\text{ís} \\
		&= \uEE{334}{3} \ulatentheatjkg \cdot 0,020 \ukilo\ugramm 
			+ 0,02 \ukilo\ugramm \cdot 4186 \uspecificheatjkgk \cdot \Delta \utemperature_\text{ís} \\
		&= 6680 \ujoule +  83,7 \uheatcapacityjk \cdot \Delta \utemperature_\text{ís}
\end{align*}
þar sem við erum bara taka stærðina af varmanum þá er krafa að hitastigsbreytingar
séu jákvæðar. Við vitum líka að ísinn fer frá $0 \udegcent$ til 
$\utemperature_\text{lok}$ og vatnið fer frá $20 \udegcent$ til $\utemperature_\text{lok}$.
Samanlagt er stærðin hitastigsbreytingu 
$\Delta \utemperature_\text{vatn} + \Delta \utemperature_\text{ís} = 20 \ukelvin$.
Sem þýðir að $\Delta \utemperature_\text{ís} = 20 \ukelvin - \Delta \utemperature_\text{vatn}$
Gefur það að varminn sem ísinn þiggur er
\begin{align*}
	\uthermaleng_\text{ís}  &= 6680 \ujoule +  83,7 \uheatcapacityjk \cdot \Delta \utemperature_\text{ís} \\
		&= 6680 \ujoule +  83,7 \uheatcapacityjk \cdot \left(20 \ukelvin - \Delta \utemperature_\text{vatn} \right)
\end{align*}
þá er hægt að setja að varminn sem vatnið gefur er þegið af ísnum
\begin{align*}
	\uthermaleng_\text{vatn} &=  \uthermaleng_\text{ís} \\
	2093 \uheatcapacityjk \cdot \Delta \utemperature_\text{vatn}
		&= 6680 \ujoule +  83,7 \uheatcapacityjk \cdot 20 \ukelvin - 83,7 \uheatcapacityjk \cdot \Delta \utemperature_\text{vatn}\\
	2093 \uheatcapacityjk \cdot \Delta \utemperature_\text{vatn}
		&= 6680 \ujoule +  1674 \ujoule - 83,7 \uheatcapacityjk \cdot \Delta \utemperature_\text{vatn} \\
	2093 \uheatcapacityjk \cdot \Delta \utemperature_\text{vatn} + 83,7 \uheatcapacityjk \cdot \Delta \utemperature_\text{vatn}
		&= 8354 \ujoule \\
	\left( 2093 \uheatcapacityjk + 83,7 \uheatcapacityjk \right) \Delta \utemperature_\text{vatn}
		&= 8354 \ujoule \\
	\left( 2177 \uheatcapacityjk \right) \Delta \utemperature_\text{vatn}
		&= 8354 \ujoule \\
	\Delta \utemperature_\text{vatn}
		&= \frac{8354 \ujoule}{ 2177 \uheatcapacityjk } \\
		&= 3,84 \ukelvin \approx 3,84 \udegcent
\end{align*}
þá er lokahitastigið fyrir vatnið og ísinn
\[
	\utemperature_\text{loka} = 20 \udegcent - 3,84 \udegcent = 16,2 \udegcent
\]
\end{formalexample}

\section{Varmafærsla}
Þegar það er hitastigsmunur í efni þá leitast hitastigið á öllum stöðum í sama
hitastigið. Þetta þýðir að innri varmi efnisins dreifist jafn um efnið, varmi
hefur ekki tilhneigingu til að halda sér á sama stað ef það er hægt að leita til
lægra hitastigs.
