% Chap units
\chapter{Einingar}
Einingar tákna stærðir, sem dæmi er vegalengd stærð en er líka mæld í mismunandi
stærðum (t.d. $\ukilo\umeter$ eða $\ucenti\umeter$. Almennt er heppilegast að
mæla í sömu einingum. Þegar mælanlegar stærðir eru gefnar á alltaf að gefa
einingu með, stærðir án eininga hafa enga merkingu. Einingakerfið sem er notað 
í eðlisfræði er kallað SI-einingakerfið, þar sem
SI stendur fyrir \emph{Systeme international d'unites} eða alþjóða einingakerfið.
\marginpar{Alþjóða einingakerfið er byggt á franska metrakerfinu, alþjóða 
einingakerfið var útgefið 1960.}

\section{SI-einingar}
Mælingar eru háðar því að geta mælt stærðir, t.d. við getum mælt vegalengd, tíma og
massa. Það er líka hægt að mæla krafta af völdum hleðslu rafeinda, mælanlegar
stærðir eru almennt tengdar við eiginleika hluta. Sem dæmi eru mælanlegar stærðir
\begin{itemize}
	\item Vegalengd
	\item Tími
	\item Massi
	\item Hleðsla
	\item Straumur
	\item Hraði
	\item Orka
\end{itemize}
sumar stærðir eru samsettar stærðir, t.d. er hraði stærð samansett af vegalengd 
og tíma. Til að hafa staðlað form af mælingum þá er alþjóða-einingakerfið samansett alltaf
af sömu einingum. Sem eru kallaðar grunneiningar sem eru gefnar í 
töflu \ref{tab:units:baseunits}.
\begin{table}[hbt]
	\label{tab:units:baseunits}
	\centering
	\begin{tabularx}{0.55\textwidth}{Xlc}
		\toprule
		Stærð & Heiti & Tákn \\
		\midrule
		Lengd & Meter & $\umeter$ \\
		\midrule
		Massi & Kílógramm & $\ukilo\ugramm$ \\
		\midrule
		Tími & Sekúnda & $\usec$ \\
		\midrule
		Rafstraumur & Amper & $\uamper$ \\
		\midrule
		Algildur hiti & Kelvin & $\ukelvin$ \\
		\midrule
		Ljósstyrkur & Kándela & $\ucandela$ \\
		\midrule
		Efnismagn & Mól & $\umol$ \\
		\bottomrule
	\end{tabularx}
	\caption{ 
		SI-grunneiningar gefnar, tákn þeirra og heiti.
		}
\end{table}
Síðan er hægt að leiða út allar aðrar einingar út frá grunneiningum, sem er stórt verk
að gera þar það finnast fjölmargar einingar. Hægt er að breyta þessum einingum í aðrar 
stærðir, svosem fet, mínútur, klukkutíma eða kílómetra. 
%
\marginpar{Fet er uþb. einn þriðji úr meter, eldri einingar voru venjulega
byggðar á mannslíkanum, t.d. faðmar og tommur}
%

Einingar sem eru samansettar úr SI-einingum er einfaldlega kallaðar samsettar einingar
og eru jafn mikið notaðar á við SI-grunneiningar (ef ekki meira), 
sjá töflu \ref{tab:units:combinedunits}. Sem dæmi er hraði samsett
eining, líka orka, og tíðni.
\begin{table}[hbt]
	\centering
	\begin{tabularx}{0.75\textwidth}{Xlcc}
		\toprule
		Stærð & Heiti & Tákn & SI-grunneining \\
		\midrule
		Kraftur & Newton & $\unewton$ & $\unewtonkgms$ \\
		\midrule
		Orka & Joule & $\ujoule$ & $\ujoulekgms$ \\
		\midrule
		Hleðsla & Couloumb & $\ucharge$ & $\uamper \usec$ \\
		\midrule
		Tíðni & Hertz & $\uhertz$ & $\uhertzs$ \\
		\bottomrule
	\end{tabularx}
	\caption{ 
		Samsettar SI-einingar, dæmi um nokkrar einingar.
		}
		\label{tab:units:combinedunits}
\end{table}
Sem hluti af SI-einingakerfinu, þá eru stöðluð forskeyti
%
\marginpar{Forskeyti eru orð (eða bókstafir) sem eru sett framan 
á önnur orð, t.d. er forskeytið
kíló á orðið gramm skrifað sem kílógramm}
%
oft nýtt til þess að gefa
til kynna stærð einingar. Þetta er stundum ákjósanlegra en að nota staðalform sem
er hentugt en getur verið illlæsilegt.
\begin{table}[!hbt]
	\centering
	\begin{tabularx}{0.35\linewidth}{@{\extracolsep{\fill}} c c c}
		\toprule
		\multicolumn{3}{c}{SI-forskeyti} \\
		\midrule
		Stærð & Tákn & Heiti\\
		\midrule
		$10^{12}$ & \utera & Tera \\
		$10^{9}$  & \ugiga & Gíga \\
		$10^{6}$  & \umega & Mega \\
		$10^{3}$  & \ukilo & Kílo \\
		$10^{2}$  & \uhekta & Hekta \\
		$10^{1}$  & \udeka & Deka \\
		\midrule
		$10^{-1}$  & \udeci & Deci \\
		$10^{-2}$  & \ucenti & Centi \\
		$10^{-3}$  & \umilli & Milli \\
		$10^{-6}$  & \umicro & Míkró \\
		$10^{-9}$  & \unano & Nanó \\
		$10^{-12}$  & \upico & Píkó \\
		\bottomrule
	\end{tabularx}
	\caption{ 
		SI forskeyti, stærð og heiti
		}
\end{table}
Til að fá tilfinningu fyrir þessum forskeytum eru sýnidæmi hentug, skal hafa 
í huga að hægt er að breyta á milli eininga. Hægt er að búa til breytistuðull
sem hentar hverri einingu að sinni, sem dæmi að breyta meter í nanómeter væri
\begin{align*}
	\si{\num{1} \nano\meter } &= \si{\num{1e-9} \meter} && \Leftrightarrow \\
	\si{\num{1e9} \nano\meter } &= \si{\num{1} \meter} && \Leftrightarrow \\
	\frac{
		\si{\num{1e9} \nano\meter }
		}{
		\si{\num{1} \meter }
		}
		&= \si{\num{1}}
\end{align*}
eða í beinum orðum er þetta $10^9$ nanómetrar á hvern meter.

\begin{formalexample}
Hvað eru $30 \si{\centi\meter}$ margir kílómetrar?
\\[4 ex]
Hægt er að fara nokkrar leiðir að þessu markmiði, fyrst er centimetrum breytt
í metra
\begin{align*}
	\num{30 } \si{ \cancel\centi\meter} \times 
		\frac{\si{\num{1 } \meter}}{\num{100} \si{ \cancel\centi\meter}} 
		&= \si{\num{0.30} \meter}
\end{align*}
síðan er metrum breytt í kílómetra
\begin{align*}
	\num{0.30} \si{\cancel\meter} \times 
		\frac{\num{1} \si{\kilo\meter}}{\num{1000}\si{\cancel\meter}} 
		&= \si{\num{0.30e-3} \kilo\meter}
\end{align*}
\end{formalexample}


\begin{formalexample}
Breyttu eftirfarandi í SI-einingar
\begin{enumerate}
	\item $20 \unano\umeter$
	\item $2 \ucenti\umeter^2$
	\item $3 \ukilo\umeter^2$
	\item $4 \umilli\ugramm$
\end{enumerate}
og sýna útreikninga.
\\[4 ex]
Þá er
\begin{enumerate}
	\item $20 \unano\umeter = 20 \cdot 10^{-9} \umeter = 2\cdot 10^{-8} \umeter$
	\item $2 \ucenti\umeter^2 = 2 \cdot \left( 10^{-2} \umeter \right)^2 = 2 \cdot 10^{-4} \umeter^2$
	\item $3 \ukilo\umeter^2 = 3 \cdot \left( 10^{3} \umeter \right)^2 = 3 \cdot 10^{6} \umeter^2$
	\item $4 \umilli\ugramm = 4 \cdot 10^{-3} \ugramm = 4 \cdot 10^{-6} \ukilo\ugramm$
\end{enumerate}
Seinasti liðurinn minnir okkur á að SI-eining fyrir massa er $\ukilo\ugramm$ eða
þúsund grömm.
\end{formalexample}

\begin{formalexample}
Breyttu eftirfarandi í SI-einingar
\begin{enumerate}
	\item $3 \frac{\ukilo\umeter}{\uhour}$
	\item $2 \frac{\ugramm}{\ucenti\umeter^3}$
	\item $5 \frac{\umicro\umeter}{\unano\usec}$
	\item $6 \frac{\umega\umeter}{\ukilo\usec}$
\end{enumerate}
og sýna útreikninga.
\\[4 ex]
Þá er
\begin{enumerate}
	\item $3 \frac{\ukilo\umeter}{\uhour} 
		= 3 \cdot \frac{10^{3}\umeter}{3600 \usec}
		= 3 \cdot \frac{1}{3,6} \frac{\umeter}{\usec}
		= 0,8 \cdot \frac{\umeter}{\usec}$
	\item $2 \frac{\ugramm}{\ucenti\umeter^3}
		= 2 \cdot \frac{10^{-3} \ukilo\ugramm}{\left( 10^{-2} \umeter \right)^3}
		= 2 \cdot \frac{10^{-3} \ukilo\ugramm}{10^{-6} \umeter^3}
		= 2 \cdot 10^{3} \frac{\ukilo\ugramm}{\umeter^3}
		$
	\item $5 \frac{\umicro\umeter}{\unano\usec}
		= 5 \cdot \frac{10^{-6}\umeter}{10^{-9}\usec}
		= 5 \cdot 10^{3} \frac{\umeter}{\usec}
		$
	\item $6 \frac{\umega\umeter}{\ukilo\usec}
		= 6 \cdot \frac{10^{6}\umeter}{10^{3}\usec}
		= 6 \cdot 10^{3} \frac{\umeter}{\usec}
		$
\end{enumerate}
\end{formalexample}

\section{Óvissa}
Þegar við vinnum með mælistærðir \index{mælistærð} þá er innbyggð
óvissa sem fylgir stærðinni. Þá er mælingin sem var gerð með
nákvæmi sem stjórnar hversu marga stafi þarf til að lýsa stærðinni.
Þá er fjöldinn af stöfunum ummerki um meiri nákvæmni, þá gilda
líka reglur um samanlagningu og margföldun varðandi slíkar tölur.
Einföld þumalputtaregla er
\begin{formalstatement}
	Minnsti fjöldi markverða stafa á reiknistærðum stjórnar fjölda
	markverða stafa í svari.
\end{formalstatement}
Lokastærðin verður þá háð því hver nákvæmnin var í upphafsmælistærðum. 
Hins vegar er líka óvissa sem fylgir mælistærð. Óvissa eru efri og neðri
mörk mælistærðar, ekki bara er hægt að tala um hvað er mesta nákvæmni
en líka hvað eru ytri þolmörk slíkrar nákvæmi. Sem dæmi er hægt að
búa til tilraun þar sem í hvert sinn er sama fjölda markverða stafa
en hins vegar breytist útkoman úr hverri mælingu. Þá þyrpast allar
mælingar umhverfis ákveðið miðgildi og hafa efri og neðir mörk. Þá þarf
að taka meðaltal af mælingum og gefa með efri og neðri mörk útfrá
þeim upplýsingum. Þá er meðaltalið af öllum mælingum
\begin{align}
	\left<{\ux}\right> &= 
		\frac{
		\text{Summa allra mælinga} 
		}{
		\text{Fjöldi mælinga}
		}
\end{align}
þar sem $\ux$ getur verið t.d. hraði, vegalengd eða aðrar stærðir. Þegar
meðaltalið er þekkt er hægt að finna ytri mörkin á mælingum. Sem er
táknað með
\begin{align}
	\left< \ux \right> \pm \text{ óvissa}
\end{align}
þess vegna er oft gefnar stærðir sem $4 \umeter \pm 0,5 \umeter$. Þar
sem meðaltal er $4 \umeter$ og óvissan er $0,5 \umeter$. Það eru til
margar leiðir til að finna efri og neðri mörk en almennt er hægt að
velja lægstu tölu úr mengi og hæðstu tölu úr mengi til að finna
efri og neðri mörk. Það er líka hægt að gefa sér stæðasta frávik 
sem bæði efri og neðri mörk.
Sumir hafa notað einfalda útgáfu af meðatali á milli neðri og efri frávika.

\begin{formalexample}
	Jóhannes fór út og ákvað að mæla stökkhraðan á engisprettum, hann
	mældi eftirfarandi \\
	\begin{center}
	\begin{tabular}{c}
		\toprule
		Hraði $\displaystyle\left[\uspeedms\right]$ \\
		\midrule
		3,11 \\
		3,0 \\
		2,92 \\
		\bottomrule
	\end{tabular}
	\end{center}
	hvað er óvissan í mælingunum?
	\\[2ex]
	Meðaltalið af öllum mælingunum er
	\[
		\left< x \right> 
			= \frac{3,11 \uspeedms + 3,0 \uspeedms + 2,92 \uspeedms}{3} 
			= 3,01 \uspeedms
	\]
	hins vegar er lægsti fjöldinn af markverðum stöfum 2 og því er ekki
	hægt að treysta meira en stærðinni $3,0 \uspeedms$. Þá eru efri og
	neðri mörk þessa mælinga uþb. $0,1 \uspeedms$ frá miðgildi. Þá er
	ágiskun á óvissu
	\[
		3,0 \uspeedms \pm 0,1 \uspeedms
	\]
\end{formalexample}

