\section*{Formáli}
Eftirfarandi efni er gefið út undir frjálsu leyfi sem er samhæft við
GPLv3%
\footnote{http://www.gnu.org/licenses/gpl-3.0.html}%
og FDL 1.3%
\footnote{http://www.gnu.org/licenses/fdl-1.3.html}%
, athugið að frjáls er hér notað undir þeim skilningi sem
er á ensku. Þá er hægt að afrita og dreifa innhaldi bókarinnar, jafnvel
leyfilegt að bæta við og breyta innihaldi eftir þörf. Eina krafan sem
fylgir því að meiga nota innhald bókarinnar í afleidd verk er að það má
ekki víkja frá frjálsu leyfi í afleiddum verkum og upphafshöfunda skal
vera getið. Skuli afleidd verk vera ekki gefin undir frjálsu leyfi er
það talið sem höfundarréttarbrot og varðar við lög. Þá telst það
refsivert undir lögum nr. 73/1972, grein 54. Refsing varðar allt að
2 árum í fangelsi og fjársektum.

Markmið þessara bókar er að reyna búa til námsefni sem er nothæft
fyrir kennslu í eðlisfræði á framhaldsskólastigi. Það eru til
fjöldi bóka en nær engar gefnar undir frjálsum leyfum. Þá er hægt
að þróa námsefni sem getur bæði verið sjálfstætt eða notað sem
viðbót við annað námsefni. Á þessu námsstigi eru fáar 
grundvallarbreytingar sem gerast í námsefni og því er ekki örar
breytingar sem gerast í efnisinnihaldi. Aftur á móti eru
bækur oft skrifaðar með málfari sem er ekki í takt við tíman,
orðabeiting og framsetning hugtaka þykja úrelt.

Hins vegar með því að bjóða frjálst efni er hægt að endurnota og
endurskrifa það með tíðarandanum. Það er ekki krafa að fá beint
leyfi frá höfundum, svo lengi sem höfundar eru nefndir og afleidd verk
haldast áfram frjáls. Þá verður lagt meira í að bæta og viðhalda
vinnunni sem fór í að skapa upphaflega verkið. Í stað þess að
hver höfundur þarf að vinna alla vinnuna uppá nýtt í hvert sinn.

Sem stendur er núverandi ritverk í mikilli vinnslu og því eru tíðar
villur og ókĺáraðir hlutar. Innihaldið sem stendur er viðbótarefni
eða önnur frammsetning á þekktu efni.
