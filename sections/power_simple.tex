% Afl
\section{Afl}
Afl er einfaldlega orka á tímaeiningu, eða í meira hefðbundnu máli, Joule á sekúndu
$\si{\J\per\s}$ sem hefur SI-eininguna Watt sem er $\si{\W}$. 
Skilgreining á afli er
\begin{equation}
	\upower = \frac{\uworkw}{\Delta\utime}
\end{equation}
almennt er $\uworkw$ vinnan sem var framkvæmd á tímabilinu $\Delta\utime$. Sem
er magnið af orku notað deilt tímanum sem það tók. Sambærilegt form af því að segja
að hraði sé hversu langt var ferðast deilt með tímanum sem það tók.
%
\begin{formalexample}
Pumpa sem dælir vatni upp úr brunni sem er $\SI{10}{\m}$ djúpur nær að dæla $\SI{5}{\m\cubed}$
á $\SI{3}{\minute}$. Hver er afl dælunnar?
\\[4 ex]
Eðlismassinn fyrir vatn er $\SI{1000}{\kg\per\m\cubed}$, sem þýðir að samanlagður massi
vatnsins er $\SI{5e3}{\kg}$, orkan sem fór lágmark í að færa vatnið upp um $\SI{10}{\m}$
er
\[
	\uworkw = \umass \uacceleg \Delta\ulengthh
		= \SI{5e3}{\kg} \times \SI{9.8}{\m\per\s\squared} \times \SI{10}{\m}
		= \SI{490}{\kJ}
\]
þá er aflið sem dælan gaf á meðan hún dældi vatninu
\[
	\upower = \frac{\uworkw}{\Delta\utime}
		= \frac{ \SI{490000}{\J} }{ \SI{180}{\s} }
		= \SI{2.7}{\kW}
\]
\end{formalexample}


\subsection{Nýtni afls}
Það er hægt að bera saman hversu vel aflið er notað til þess sem áætlað er, hversu góð
nýtingin er táknuð með stærðinni $\eta$. Þetta svipar til núningsstuðulsins, er 
prósentuhlutfallið sem er hægt að nýta. Nýtni er gefin við
\begin{equation}
	\eta = \frac{\upower_\text{út}}{\upower_\text{inn}}
\end{equation}
þá er átt við að $\upower_\text{inn}$ er aflið gefið og $\upower_\text{út}$ er aflið sem
fór í að áætlað verk. Ber að athuga að nýtni liggur á milli $0$ og $1$, þ.e.a.s. $0 \%$ og
$100\%$ nýtingu.

\begin{formalexample}
Mótor sem dregur upp lyftu með farþegum er skráður til að hafa aflið $\SI{8}{\kW}$,
samanlagður massi lyftu og farþega er $\SI{800}{\kg}$ og mótorinn dregur lyftuna upp
$\SI{12}{\m}$ á $\SI{15}{\s}$. Hver er nýtni mótorsinns
\\[4 ex]
Aflið sem fer í að hífa lyftuna séð út frá vélrænni orku er
\[
	\upower_\text{út} 
		= \frac{\umass \uacceleg \Delta\ulengthh}{\Delta\utime}
		= \frac{ \SI{800}{\kg} \times \SI{9.8}{\m\per\s\squared}
			\times \SI{12}{\m} }{ \SI{15}{\s} }
		= \SI{6272}{\W} \approx \SI{6.3}{\kW}
\]
sem gefur að nýtni mótorsinns til að hífa lyftuna upp er
\[
	\eta = \frac{ \SI{6.3}{\kW} }{ \SI{8}{\kW} } 
		= \num{0.79}
\]
þ.e.a.s. það fer $79\%$ af aflinu frá mótorinnum í að toga lyftuna upp á meðan $21 \%$ af aflinu
fer mótstöðukrafta eða núning.
\end{formalexample}
