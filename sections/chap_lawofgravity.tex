% Chap wave motion
\chapter[Þyngdarlögmál]{Þyngdarlögmál Newtons}
Hlutir með massa verka á hvorn annan með þyngdarkrafti, þar á meðal verka
plánetur á sólina og sólin á plánetur. Nánar tiltekið massar verka jafnt á
hvorn annan. Það er ekki ennþá búið að útskýra afhverju massar geta
verkað með þyngdarkrafti en það er búið að lýsa því \emph{hvernig}
mjög vel. Eitt af efnisgreinum Newtons var að lýsa því hvernig
plánetur hreyfast undir áhrifum þyngdarkrafta. Sem hluti af þeirri lýsingu
þarf að koma með fasta sem er kallaður þyngdarfastinn $\ugravconstant$.
Stærðin sem Newton kom er
\begin{equation}
	\uforce = \ugravconstant \frac{\umassm \umassM}{\ulengthr^2}
\end{equation}
þeas. krafturinn milli tveggja massa breytist í öfugri ferningsstærð
vegalengdarinnar. Það er mikið til að byrja með, en þetta er
óvenju saklaus jafna þegar uppi er staðið.

\begin{formalexample}
Tveir loftsteinar eru á floti í geimnum, annar hefur massan 2 tonn og hinn hefur
massan 4 tonn, hver er þyngdarkrafturinn á milli þeirra þegar þeir eru $3 \ukilo\umeter$
frá hvor öðrum?
\\[4 ex]
Þá er krafturinn á milli steinana
\begin{align*}
	\uforce &= \ugravconstant \frac{\umassm \umassM}{\ulengthr^2} \\
		&= \ugravconstantnmkg \cdot \frac{ 2000 \ukilo\ugramm \cdot 3000 \ukilo\ugramm}{\left( 3000\umeter \right)^2} \\
		&= \uEE{4,45}{-11} \unewton
\end{align*}
\end{formalexample}

\begin{formalexample}
Hver er krafturinn sem jörðin verkar á $50 \ukilo\ugramm$ persónu við yfirborð
jarðar?
Radíus jarðar er $\uEE{6.37}{6} \umeter$ og massi jarðar er $\uEE{6}{24} \umeter$.
\\[4 ex]
Þá er krafturinn jörðin verkar með
\begin{align*}
	\uforce &= \ugravconstant \frac{\umassm \umassM}{\ulengthr^2} \\
		&= \ugravconstantnmkg 
			\cdot \frac{ 50 \ukilo\ugramm \cdot 
				\uEE{5,97}{24} \ukilo\ugramm}{\left( \uEE{6,731}{6} \umeter \right)^2} \\
		&= 490 \unewton
\end{align*}
\end{formalexample}

\section{Stefna þyngdarkrafts}
Lögmál Newtons gefur til um stærð þyngdarkraftsins en hins vegar kemur ekki með
lýsingu á rúmlegri uppsetningu. Þá er ekki mikið sagt annað en
að krafturinn aðdráttarkraftur. Þá verkar þyngdarkraftur á báða hlutina
og báðir togast að hvor öðrum. Til að lýsa því er hægt að bæta við
vigur formi af þyngdarlögmálinu
\begin{equation}
	\bar\uforce
		= \ugravconstant \frac{\umassm \umassM}{\left|\ulengthr\right|^2} \hat \ulengthr
\end{equation}
þar sem $\hat \ulengthr$ er stefnuvigur af stærðinni $1$ sem miðar alltaf í átt
að gagnstæðum massa og $\left|\ulengthr\right|$ er vegalengdin á milli massanna.
Síðan er hægt að hafa marga massa sem verka á einn hlut, þá þarf fara út í
nokkuð langa útreikininga.

\begin{formalexample}
Á milli tveggja loftsteina er massi $\umassm$ sem er $20 \ukilo\ugramm$, loftsteinn
A hefur massann $\umassM_\text{A} = 3000 \ukilo\ugramm$ og loftsteinn B hefur massann
$\umassM_\text{B} = 2000 \ukilo\ugramm$. Vegalengdin á milli A og B er 
$\ulengthr_\text{A} = 1500 \umeter$ og $\ulengthr_\text{B} = 2000 \umeter$.
Hver er krafturinn sem verkar á massan $\umassm$
\\[4 ex]
Þá er krafturinn á milli loftsteina A og massa
\begin{align*}
	\uforce_\text{A,m} &= \ugravconstant \frac{\umassm \umassM}{\ulengthr^2} \\
		&= \ugravconstantnmkg \cdot \frac{ 20 \ukilo\ugramm \cdot 3000 \ukilo\ugramm}{\left( 1500\umeter \right)^2} \\
		&= \uEE{1,78}{-12} \unewton
\end{align*}
og á milli loftsteins B og massans
\begin{align*}
	\uforce_\text{B,m} &= \ugravconstant \frac{\umassm \umassM}{\ulengthr^2} \\
		&= \ugravconstantnmkg \cdot \frac{ 20 \ukilo\ugramm \cdot 2000 \ukilo\ugramm}{\left( 2000\umeter \right)^2} \\
		&= \uEE{6,67}{-13} \unewton
\end{align*}
þá er heildarkrafturin sem verkar á massan
\begin{align*}
	\uforce_\text{heild} &= -\uforce_\text{A,m} + \uforce_\text{B,m} \\
		&= -\uEE{1,78}{-12} \unewton + \uEE{6,67}{-13} \unewton \\
		&= -\uEE{1,11}{-12} \unewton
\end{align*}
neikvætt formerki hér táknar að krafturinn sem verkar á massan stefni á loftstein A.
\end{formalexample}
\begin{formalexample}
Þrír massar (A, B og C) eru allir í láréttu plani, og mynda jafnhliða þríhyrning. Hliðar
þríhyrningins eru $5 \ukilo\umeter$, massar A og B eru $4000 \ukilo\ugramm$ á meðan
massi C er $6000 \ukilo\ugramm$. Hver er stefna og stærð kraftsins sem verkar á
massa C?
\\[4 ex]
Valið er að setja massan C í toppinn á þríhyrningnum, þá er heildarkrafturinn sem
verkar á C
\[
	\bar\uforce_\text{C} = \bar\uforce_\text{A,C} + \bar\uforce_\text{B,C}
\]
þyngdarkrafturinn á milli A og C er
\begin{align*}
	\bar\uforce_\text{A,C} &= \ugravconstantnmkg \cdot
		\frac{4000 \ukilo\ugramm \cdot 6000 \ukilo\ugramm}{\left|5000 \umeter\right|^2} \hat \ulengthr_\text{A,C} \\
	&= \uEE{6,40}{-11} \unewton \cdot 
		\begin{bmatrix}
			-\cos(30 \udeg) \\[0.3em]
			-\sin(30 \udeg) 
		\end{bmatrix} \\
	&=	\begin{bmatrix}
			- \uEE{5,54}{-11} \unewton \\[0.3em]
			- \uEE{3,20}{-11} \unewton
		\end{bmatrix}
\end{align*}
þyngdarkrafturinn á milli B og C er
\begin{align*}
	\bar\uforce_\text{B,C} &= \ugravconstantnmkg \cdot
		\frac{4000 \ukilo\ugramm \cdot 6000 \ukilo\ugramm}{\left|5000 \umeter\right|^2} \hat \ulengthr_\text{B,C} \\
	&=	\begin{bmatrix}
			\uEE{5,54}{-11} \unewton \\[0.3em]
			- \uEE{3,20}{-11} \unewton
		\end{bmatrix}
\end{align*}
sem gefur heildarkraftinn
\begin{align*}
	\bar\uforce_\text{C} &= \bar\uforce_\text{A,C} + \bar\uforce_\text{B,C} \\
		&= 
		\begin{bmatrix}
			- \uEE{5,54}{-11} \unewton \\[0.3em]
			- \uEE{3,20}{-11} \unewton
		\end{bmatrix}
		+
		\begin{bmatrix}
			\uEE{5,54}{-11} \unewton \\[0.3em]
			- \uEE{3,20}{-11} \unewton
		\end{bmatrix} \\
		&= 
		\begin{bmatrix}
			- \uEE{5,54}{-11} \unewton + \uEE{5,54}{-11} \unewton \\[0.3em]
			- \uEE{3,20}{-11} \unewton - \uEE{3,20}{-11}  \unewton
		\end{bmatrix}\\
		&= 
		\begin{bmatrix}
			0 \unewton \\[0.3em]
			- \uEE{6,40}{-11} \unewton
		\end{bmatrix}
\end{align*}
sem þýðir að stærð kraftsins er
\[
	\left|\bar\uforce_\text{C} \right| 
		= \sqrt{\left(0 \unewton\right)^2 + \left(-\uEE{6,40}{-11} \unewton\right)^2}
		= \uEE{6,40}{-11} \unewton
\]
\end{formalexample}


\section{Þyngdarsvið}
Skilgreining á þyngdarsviði er kraftur per kílógramm, við yfirborð jarðar svarar
það til þyngdarhröðunar
\begin{equation}
	\bar\ugravfield =  \frac{\uforce}{\umassm} 
		= \ugravconstant \frac{\umassM}{\left|\ulengthr\right|^2} \hat \ulengthr
\end{equation}
einingin fyrir þyngdarsvið er hröðun, en hins vegar þykir það betra að nota
$\frac{\unewton}{\ukilo\ugramm}$ til að hafa lýsingu sem tengist kraftinum sem
verkar á massann. Nánar tiltekið er massinn kallaður prufumassi, sem hugtak er
það massi sem verður fyrir verkun annara massa án þess að breyta
skipulagi kerfisins. Seinna meir mun hugtakið prufuhleðsla koma á sama máta
fram til lýsa krafti per hleðslu sem verkar á hlut.

Annar eiginleiki er að hægt er að leggja saman þyngdarsvið á sama máta eins
og hægt er að leggja saman krafta sem verka á einn hlut. Í einvíðu samhengi
þá er það fremur fljótgert en þegar komið er upp í tvær eða fleiri
rúmvíddir verður reiknivinnan talsvert þyngri.

\section{Orka í þyngdarsviði}
Aukin í stöðuorku er einfaldlega heildin af kraftinum yfir þá vegalengd sem
hluturinn færist yfir. Almennt er hægt að segja vinnan sé
\[
	\uworkw = \int_{\ulengthr_1}^{\ulengthr_2} \uforce d\ulengthr
\]
eða gefur að
\begin{align*}
	\uworkw &= \int_{\ulengthr_1}^{\ulengthr_2} 
		\ugravconstant \frac{\umassm \umassM}{\ulengthr^2} d\ulengthr \\
		&= \ugravconstant \umassm \umassM 
			\int_{\ulengthr_1}^{\ulengthr_2} \frac{d\ulengthr}{\ulengthr^2} \\
		&= \ugravconstant \umassm \umassM 
			\left[ - \frac{1}{\ulengthr} \right]_{\ulengthr_1}^{\ulengthr_2} \\
		&= \ugravconstant \umassm \umassM 
			\left( \frac{1}{\ulengthr_1} - \frac{1}{\ulengthr_2} \right)
\end{align*}
til að lýsa því hversu mikla orku þarf til að losa sig algjörlega undan áhrifum
þyngdarsviðs frá hlut þá er hægt að láta byrjunar punkti vera óendanlega langt
frá upphafspunt og síðan verður massinn $\umassm$ færður nær og nær þar til hann
nær stöðunni $\ulengthr$ sem gefur
\begin{align*}
	\uworkw &= \ugravconstant \umassm \umassM 
			\left( \frac{1}{\ulengthr_1} - \frac{1}{\ulengthr_2} \right) \\
		&= \ugravconstant \umassm \umassM 
			\left( \frac{1}{\infty} - \frac{1}{\ulengthr} \right) \\
		&\approx -\frac{\ugravconstant \umassm \umassM}{\ulengthr}
\end{align*}
neikvætt formerki hérna táknar að massinn hefur tapað orkunni í fallinu en hins
vegar þá myndi það kost nákvæmlega jafn mikið af orku til að komast óendanlega
langt frá stöðunni $\ulengthr$. Hugmyndin verður þá að miðað við núllpunkt sem
er óendanlega langt frá er hægt að finna hlutfallslega breytingu í stöðuorku miðað
við óendanlega punktinn. Tap hér, er tap af stöðuorku hlutarins.

\begin{formalexample}
Hver er stöðuorka hlutars sem er $400 \ukilo\umeter$ yfir jörðu og hefur
massann $200 \ukilo\ugramm$? Áætla má að hluturinn hafi byrjað við yfirborð jarðar.
\\[4 ex]
Þá er aukningin í stöðuorku miðað við núll punkt í óendanlegu
\begin{align*}
	\Delta \upotentialu &= \upotentialu_2 - \upotentialu_1 \\
		&= -\frac{\ugravconstant \umassm \umassM}{\ulengthr_2}
			+ \frac{\ugravconstant \umassm \umassM}{\ulengthr_1} \\
		&= -\frac{\ugravconstant \umassm \umassM}{\ulengthr_2}
			+ \frac{\ugravconstant \umassm \umassM}{\ulengthr_1} \\
		&= \ugravconstant \umassm \umassM \left(-\frac{1}{\ulengthr_2}
			+ \frac{1}{\ulengthr_1}  \right) \\
		&= \ugravconstantnmkg \cdot 200 \ukilo\ugramm \cdot \uEE{6,0}{24} \ukilo\ugramm \\
		&\ldots \cdot \left(-\frac{1}{\uEE{6,77}{6} \umeter}
			+ \frac{1}{\uEE{6,37}{6} \umeter}  \right) \\
		&= \uEE{7,42}{8} \ujoule
\end{align*}
Það er mun heppilegra að vinna núllpunkt í óendanlegu einmitt uppá að
geta skoða breytingar án þess að miða við miðju þyngdarsviðins.
\end{formalexample}
