% Chap projectile motion
\chapter{Kasthreyfing}
Í fyrri köflum hefur verið skoðað hvernig hlutir falla í frjálsu falli, núna er markmiðið
að skoða hvernig hlutur hagar sér í skákasti. Það er fyrst að skoða hvernig krafturinn
sem verkar á hlut er í frjálsu falli. Hann er
\[
	\bar\uforce 
		= 
		\umass
		\left( 
			\begin{array}{c}
				0 \\
				- \uacceleg \\
			\end{array}
		\right)
\]
s.s. það er enginn kraftur sem reynir að toga hlutinn áfram á meðan hann er í skákastinu.
Fyrir svona hlut er heildarkrafturinn einungis þyngdarkraftur, sem gefur
\[
	\bar\uforce_\text{heild}
		= 
		\umass
		\left( 
			\begin{array}{c}
				0 \\
				- \uacceleg \\
			\end{array}
		\right)
		=
		\umass
		\bar\uaccelea
	\Leftrightarrow
	\bar\uaccelea
		= 
		\left( 
			\begin{array}{c}
				0 \\
				- \uacceleg \\
			\end{array}
		\right)
\]
Þá er hægt að skrifa hraða hlutarins meðfram sitthvorum ásnum til að vera
\[
	\bar\uspeed 
		= 
		\left( 
			\begin{array}{c}
				\uspeed_\text{x} \\
				\uspeed_\text{y} - \uacceleg \utime \\
			\end{array}
		\right)
\]
þar sem $\uspeed_\text{x}$ og $\uspeed_\text{y}$ eru upphafshraðar meðfram x og y ásum
hnitakerfisins. Áframhaldið er að vita hver er staðsetning hlutarins eftir ákveðinn
langan tíma. Sem gefur að
\[
	\bar\ulengths 
		= 
		\left( 
			\begin{array}{c}
				\uspeed_\text{x} \utime \\
				\uspeed_\text{y} \utime - \frac{1}{2} \uacceleg \utime^2 \\
			\end{array}
		\right)
\]
s.s. út kemur að staðsetning hlutarins er einungis háð tíma
sem hefur liðið frá upphafspunkti. Í meira almennu formi er hægt að bæta við færslu
frá upphafstaðsetningu. Sem gefur
\begin{align}
	\bar\ulengths 
		= 
		\left( 
			\begin{array}{c}
				\uspeed_\text{x} \utime \\
				\uspeed_\text{y} \utime - \frac{1}{2} \uacceleg \utime^2 \\
			\end{array}
		\right)
	+
	\bar\ulengths_0
\end{align}
þar sem $\bar\ulengths_0$ er færsla frá upphafsstaðsetningu hnitakerfisins. Síðan
er hægt að setja upp ýmsar tilraunir sem fiina upphafshraða og
stefnu hlutarins í kastihreyfingunni. Líka er hægt að skrifa sama vigraform
sem sett af jöfnum
\begin{align}
	\ux(\utime) &= \ulengths_\text{x} + \uspeed_\text{x} \utime \\
	\uy(\utime) &= \ulengths_\text{y} + \uspeed_\text{y} \utime - \frac{1}{2} \uacceleg \utime^2
\end{align}
þar sem $\ulengths_\text{x}$ og $\ulengths_\text{y}$ tákna færslu frá upphafstaðsetningu
meðfram sitthvorum ás.

\begin{formalexample}
Kúlu er skotið fram af borði með lárétta hraðanum $6 \uspeedms$, hæð borðsins yfir gólfinu er
$1,2 \umeter$. Hversu langt lendir kúlan frá borðinu?
\\[4 ex]
Til að byrja með þá er nauðsyn finna tímann sem tekur fyrir kúluna að lenda á gólfinu
til að finna hversu langt hún lendir frá borðinu. Fyrir y-ásinn þá er  
\[
	\uy(\utime) = 0 \uspeedms \cdot \utime - \frac{1}{2} \uacceleg \utime^2
		= - \frac{1}{2} \uacceleg \utime^2
\]
og þegar kúlan lendir á gólfinu er staðsetningin meðfram y-ásnum
\[
	-1,2 \umeter = - \frac{1}{2} \uacceleg \utime^2 
		\Leftrightarrow
		\utime = \sqrt{ \frac{2 \cdot (1,2 \umeter)}{\uacceleg}}
			= \sqrt{ \frac{2 \cdot (1,2 \umeter)}{ 9,8 \uaccelems}}
			= 0,49 \usec
\]
þá er staðsetningin meðfram x-ásnum er 
\[
	\ux(0,49 \usec) = 6 \uspeedms \cdot 0,49 \usec = 2,94 \umeter
\]
\end{formalexample}
Það er hægt að umskrifa fyrri jöfnur sem fall af stærð upphafshraða $\uspeed$
og hornsins $\theta$ sem hlutnum er skotið af stað með. Þá eru jöfnurnar sem
eru nýttar (miðað við núll færslu frá núllpunkt)
\begin{align*}
	\ux(\utime) &= \uspeed_\text{x} \utime \\
	\uy(\utime) &= \uspeed_\text{y} \utime - \frac{1}{2} \uacceleg \utime^2
\end{align*}
hægt er að einangra tímann til að vera
\[
	\ux(\utime) = \ux = \uspeed_\text{x} \utime 
		\Leftrightarrow
		\utime = \frac{\ux}{\uspeed_\text{x}}
\]
innsetning gefur
\begin{align*}
	\ux \left( \frac{\ux}{\uspeed_\text{x}} \right) &= \uspeed_\text{x} \frac{\ux}{\uspeed_\text{x}} \\
	\uy \left( \frac{\ux}{\uspeed_\text{x}} \right) &= \uspeed_\text{y} \frac{\ux}{\uspeed_\text{x}} 
		- \frac{1}{2} \uacceleg \left( \frac{\ux}{\uspeed_\text{x}} \right)^2
\end{align*}
þá dettur jafnan sem lýsir hreyfingu meðfram x-ásnum út en jafnan sem er meðfram
y-ásnum er
\begin{align*}
	\uy &= \uspeed_\text{y} \frac{\ux}{\uspeed_\text{x}} 
		- \frac{\uacceleg}{2}  \left( \frac{\ux}{\uspeed_\text{x}} \right)^2 \\
	\uy	&= \frac{\uspeed_\text{y}}{\uspeed_\text{x}} \ux
		- \frac{\uacceleg}{2} \frac{\ux^2}{\uspeed_\text{x}^2} \\
	\uy	&= \frac{\uspeed \sin(\theta)}{\uspeed \cos(\theta)} \ux
		- \frac{\uacceleg \ux^2}{2\uspeed^2 \cos^2(\theta)} \\
	\uy	&= \tan(\theta) \ux
		- \frac{\uacceleg \ux^2}{2\uspeed^2 \cos^2(\theta)}
\end{align*}
sem gefur að hlutur sem hefur upphafshraða og horn hefur ferilinn með $\ux$
sem breytur fyrir fallið $\uy(\ux)$
\begin{equation}
	\uy(\ux) = \tan(\theta) \ux
		- \frac{\uacceleg \ux^2}{2\uspeed^2 \cos^2(\theta)}
\end{equation}
Þá er hægt að búa til sýnidæmi úr þessu
\begin{formalexample}
Kúlu er kastað með hraðanum $25 \uspeedms$ á horninu $30 \udeg$, það er 
$3 \umeter$ hár veggur sem er $40 \umeter$ langt í burtu frá kaststaðsetningunni.
Kemst kúlan yfir vegginn?
\\[4 ex]
Þar sem við þekkjum hraða, horn og vegalengd er það fljótt mál að finna hæðina
sem kúlan er
\begin{align*}
	\uy(40 \umeter)	&= \tan(30 \udeg) \cdot 40 \umeter
		- \frac{9,8 \uaccelems \cdot \left(40 \umeter\right)^2}{2 \left( 25 \uspeedms \right)^2 \cos^2(30 \udeg)} \\
		&= 23,5 \umeter
\end{align*}
sem er talsvert hærra en veggurinn sem er $3 \umeter$ hár. Þá kemst kúlan yfir.
\end{formalexample}

\section{Hámarks kastvegalengd}
Markmiðið hér að sýna hvað er hornið sem gefur mestu vegalengdina fyrir ákveðinn
upphafshraða.
\[
	\uy(\ux) = \tan(\theta) \ux
		- \frac{\uacceleg \ux^2}{2\uspeed^2 \cos^2(\theta)}
\]
til að einfalda uppsetninguna þá er bara skoðað þegar hluturinn lendir aftur
á upphafspunkt vegalengdina $\ux$ frá núllpunkti
\[
	0 = \tan(\theta) \ux
		- \frac{\uacceleg \ux^2}{2\uspeed^2 \cos^2(\theta)}
\]
sem er hægt að umskrifa sem
\begin{align*}
	0 &= \tan(\theta) \ux
		- \frac{\uacceleg \ux^2}{2\uspeed^2 \cos^2(\theta)} \\
	\frac{\uacceleg \ux^2}{2\uspeed^2 \cos^2(\theta)} 
		&= \tan(\theta) \ux \\
	\frac{\uacceleg \ux}{2\uspeed^2 \cos^2(\theta)} 
		&= \tan(\theta)  \\
	\frac{\uacceleg \ux}{2\uspeed^2 \cos^2(\theta)} 
		&= \frac{\sin(\theta)}{\cos(\theta)}  \\
	\frac{\uacceleg \ux}{2\uspeed^2} 
		&= \sin(\theta)\cos(\theta)  \\
	\uacceleg \ux
		&= 2\uspeed^2\sin(\theta)\cos(\theta)  \\
	\ux
		&= \frac{2\uspeed^2\sin(\theta)\cos(\theta)}{\uacceleg}
\end{align*}
sem í fljótu bragði virðist ekki vera sérstaklega hagnýtin lýsing á hámarksvegalengdinni.
Hins vegar er hægt að umskrifa hornaföllin sem
\[
	2\sin(\theta)\cos(\theta) = \sin(2\theta)
\]
sem einfaldar fyrri jöfnu til að vera
\begin{align}
	\ux
		&= \frac{\uspeed^2 \sin(2\theta)}{\uacceleg}
\end{align}
þá er hámarksvegalengd náð þegar breytingin á $\ux$ staðsetningu er núll á meðan
hornið breytist, sem gefur
\begin{align*}
	0 
		&= \frac{ d\ux }{d\theta} \\
	0
		&= \frac{ d }{d\theta} \left(\frac{\uspeed^2 \sin(2\theta)}{\uacceleg} \right) \\
	0
		&= \frac{\uspeed^2}{\uacceleg} \frac{ d\sin(2\theta) }{d\theta} \\
	0
		&= \frac{\uspeed^2}{\uacceleg} 2\cos(2\theta)\\
	0
		&= \frac{2\uspeed^2}{\uacceleg} \cos(2\theta)
\end{align*}
sem getur bara verið jafnt og núll ef
\begin{align*}
	0
		&= \frac{2\uspeed^2}{\uacceleg} \cos(2 \cdot 45\udeg)\\
	0
		&= \frac{2\uspeed^2}{\uacceleg}  \cdot 0\\
	0
		&= 0\\
\end{align*}
s.s. við hornið $45\udeg$ uppfyllist skilyrðið að $\frac{ d\ux }{d\theta}$ er núll.
Stærðin $\frac{ d\ux }{d\theta}$ segir hver er breytingin á staðsetningu miðað
við breytingu á horni. Á einhverju horni verður sú stærð jafnt og núll, þar sem
hún er jákvæð á meðan við aukum vegalengdina sem hluturinn kemst og síðan aftur
neikvæð þegar hornið verður of stórt og við minnkum vegalengdina sem hluturinn
kemst.