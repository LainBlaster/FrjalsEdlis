\section{Óreiða}
Allt efni hefur innbyggða orku sem tengist hita stigi efnisins, sé hitastigið
hærra verður orkan hærri. Samtímis er hægt að koma með nýtt hugtak, sem er að
efnið hefur innbyrðis röðun og getur bara haft ákveðið mikið af uppröðunum
af staðsetningu og orku. Fyrst er þetta hugtak mjög sérstakt því óreiða er
eitthvað sem öll efni reyna að hámarka, þá vill efni koma sem mestu ,,óskipulagi''
í umhverfið. Óreiðu er líka hægt að lýsa sem magn af orku sem á eftir að
ná að dreifa sér út almennilega. Það eru talsvert margar myndlíkingar
á óreiðu en það er hægt að lýsa magninu af óreiðu með tölum og
stærðum. Djúp lýsing á óreiðu sem nær alla leið niður á atómskalan krefst
umtalsverðar stærðfræði og því verður ekki meðhöndluð á þann máta. 
Verður stuðst við myndlíkingar til að reyna skapa skilning.