% Chap temperature
\chapter{Hiti}
Hitastig er lýsing á hreyfingu atóma, eftir því sem hitastigið er hærra þá er
hreyfingin meiri. Í daglegu tali er oftast notast við Celcíus skalann ($\udegcent$),
hins vegar finnast margar leiðir til að kvaðra hita. Hiti er mælieining sem er
miðuð við einhvern kvaðra, t.d. er Celcíus kvarðinn miðaður við bræðslumarks
vatns. Farenheit kvarðinn er miðaður við jafnvægisblöndu af þrem efnum,
vatni, ís og ammóníumklóri. Kelvinn kvarðinn miðast við meðalhreyfiorku atóma
og núll hreyfiorka er jafngildi núll hita. Einingin fyrir Kelvin er $\ukelvin$.
Til að breyta kelvin í gráður celcíus
\begin{equation}
	\utemperature_\text{celcíus} = \utemperature_\text{kelvin} - 273,15
\end{equation}
og til að breyta gráður celcíus í kelvin
\begin{equation}
	\utemperature_\text{kelvin} = \utemperature_\text{celcíus} + 273,15
\end{equation}
sem er hægt að nýta á marga máta. Það ætti að vera venja að alltaf vinna með
SI-einingar. Fyrir flestar jöfnur er það krafa að vinna með Kelvin skalann,
undanteikningin er þegar breyting á hitastigi á sér stað. Fyrir Kelvin og Celcíus
þá er $\Delta\utemperature_\text{kelvin} = \Delta\utemperature_\text{celcíus}$.

\begin{formalexample}
Breyttu $92 \udegcent$ í kelvin og breyttu $310 \ukelvin$ í gráður celcíus.
\\[4 ex]
Þá er
\[
	92 \udegcent = (92+273,15) \ukelvin = 365 \ukelvin
\]
og
\[
	310 \ukelvin = (310-273,15) \udegcent = 36,8 \udegcent \approx 37 \udegcent
\]

\end{formalexample}