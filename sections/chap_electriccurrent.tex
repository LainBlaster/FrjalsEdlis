\chapter{Straumur}
% Intro to chapter, laws of force
Þegar rafeindir ferðast í leiðurum þá myndast rafstraumur, eða einfaldlega
straumur. Vatnsstraumur er t.d. vatnsmagn á tímaeiningu, oft lýst sem
massi per tímaeingu $\frac{\ukilo\ugramm}{\usec}$. Á svipaðan máta 
er straumur gefinn sem $\frac{\ucoulomb}{\usec}$, þar sem $\ucoulomb$ er
,,magnið'' af rafmagni (svipar til massa). Einingin sem er notuð fyrir
straum er Amperé\footnote{Fengið frá
\url{http://en.wikipedia.org/wiki/Andr\%C3\%A9-Marie_Amp\%C3\%A8re}}
($\uamper$) sem er það sama og að segja $\uampercs$.
\marginpar{
	\begin{center}
	\includegraphics*[
		width= \marginparwidth
		]{./pictures/Ampere_Andre_1825.eps}
	\end{center}
	André-Marie Amperé(1775-186) var franskur 
	vísindamaður sem gerði tilraunir með
	rafkrafta í rafmagnsleiðurum.
	}


\section{Hleðsla og rafeindir}
% Examples of size, direction
Hleðsla sem stærð er eitthvað sem sumar eindir búa yfir sem eiginleiki,
þetta er svipað og sumar eindir hafa massa (t.d. $\ukilo\ugramm$). Það
eru ekki allar eindir sem búa yfir massa, ljós er massalaus eind á meðan
rafeindin hefur massa. Hleðsla er ,,form'' af massa, eitthvað sem sumar
eindir hafa og aðrar ekki, í þessum kafla eru bara rafeindir, jáeindir
og róteindir með hleðslu. Það eru talsvert fleiri eindir sem bera hleðslu
á sama máta og það eru til talsvert fleiri eindir sem hafa massa. Þá
er \emph{grunnhleðsla} hugtak sem er viðmiðunarpunktur af hleðslu.
Grunnhleðslan er
\begin{align}
	\uelementcharge &= \uEE{1,602}{-19} \ucoulomb
\end{align}
sem er jafn stór og hleðslan og ein rafeind, jáeind eða róteind hefur.
\begin{table}[!h]
	\begin{center}
	\begin{tabular}{ccc}
	\toprule
	Heiti & Tákn & Hleðsla \\
	\midrule
	Rafeind  & $\uelectron$ & $-\uelementcharge$ \\
	Jáeind  & $\upositron$ & $+\uelementcharge$ \\
	Róteind  & $\uproton$ & $+\uelementcharge$ \\
	\bottomrule
	\end{tabular}
	\end{center}
	\caption{Yfirlit yfir hleðslu frumeinda}
\end{table}
\begin{formalexample}
Hvað eru margar rafeindir í einu kúlombi?
\\[4 ex]
Þegar grunnhleðslan per rafeind er
\[
	\uelementcharge = \uEE{1,602}{-19} \frac{\ucoulomb}{\text{rafeind}}
\]
þá er fjöldi rafeinda
\begin{align*}
	\text{fjöldi} =
		\frac{ 1 \ucoulomb}{\uEE{1,602}{-19} \frac{\ucoulomb}{\text{rafeind}} }
	= \uEE{6,242}{18} \text{rafeindir}
\end{align*}
sem er talsverður fjöldi rafeinda. Almennt er eitt $\ucoulomb$ fremur stór stærð
af rafeindum en þegar við mælum straum er hún hentugri til en fjöldi rafeinda.
\end{formalexample}

\section{Straumur}
% Current
Straumur er þegar hleðsla (sem geta verið rafeindir) færist á milli staða, færsla
á hleðsu myndar strauminn en annað hugatak sem kallast spenna sér um að ýta
hleðslunni áfram. Til samanburðar er hægt að ímynda sér að hleðsa sé vatn og
þyngdarkrafturinn sé spennan sem togar vatnið áfram. Einingin fyrir straum er
$\uamper$ eða $\uampercs$, SI-einingin er samt $\uamper$ er oftast gefinn upp.
Þá er straumur gefinn við
\begin{align}
	\ucurrenti &= \frac{\Delta\uchargeQ}{\Delta\utime}
\end{align}
þar sem er $\Delta\uchargeQ$ magnið af hleðslu sem hefur fengið færslu á 
tímabilinu $\Delta\utime$.
\begin{formalexample}
Mjög þolinmóður talningsmaður hefur talið að í $1\ugramm$ af kopari eru
$\uEE{1,5}{20}$ rafeindir sem geta fært sig. Hann tengir vír við 
koparinn (undir spennu) og mælir strauminn $5 \umilli\uamper$. Hvað tekur það
langan tíma að færa allar rafeindirnar úr koparinum?
\\[4 ex]
Þegar grunnhleðslan per rafeind er
\[
	\uelementcharge = \uEE{1,602}{-19} \frac{\ucoulomb}{\text{rafeind}}
\]
þá er heildarhleðslan
\begin{align*}
	\Delta\uchargeQ &=
		\uEE{1,602}{-19} \frac{\ucoulomb}{\text{rafeind}} \cdot
		\uEE{1,5}{20} \text{rafeindir}
	&= 24,03 \ucoulomb
\end{align*}
hægt er að umskrifa straumjöfnuna til að vera
\begin{align*}
	\Delta\utime &= \frac{\Delta\uchargeQ}{\ucurrenti}
		&= \frac{24,03 \ucoulomb}{\uEE{5}{-3} \uamper}
		&= 4806 \usec = 80 \umin
\end{align*}
ef spennan yrði aukin myndi tíminn minnka, samsvarar að nota stærri kraft til að
hreyfa rafeindirnar.
\end{formalexample}
