%
\section{Þverkraftur}
Ef við látum kraft verka á flöt er krafturinn sem verkar beint út frá fletinum kallaður
þverkraftur, þá er stefna kraftsins hornrétt á flötinn. Krafturinn sem er notaður til að
skilgreina hvað þrýstingur er
\begin{equation}
	\upressure = \frac{\uforce_\text{þver}}{\uarea}
\end{equation}

\begin{formalexample}
Maður með massann $\SI{80}{\kg}$ stendur á plötu sem er $\SI{0.5}{\m}$ á breidd og
$\SI{0.2}{\m}$ á lengd. Hver er þrýstingurinn sem botninn á 
plötunni upplifir vegna mannsins?
\\[4 ex]
Þyngdarkrafturinn vegna mannsins er
\[
	\uforce_\uacceleg = \umass \uacceleg 
		= \SI{80}{\kg} \times \SI{9.8}{\m\per\s\squared}
		= \SI{784}{\N}
\]
þar sem krafturinn verkar hornrétt á plötuna og það er kraftjafnvægi, þá er 
$\uforce_\uacceleg = \uforce_\text{þver}$ sem gefur
\[
	\upressure = \frac{ \SI{784}{\N} 
			}{ \SI{0.5}{\m} \times \SI{0.2}{\m} }
		= \SI{7840}{\Pa}
\]
\end{formalexample}


\section{Þyngdarþrýstingur}
% Pressure in fluid at depth
Þegar maður fer dýpra í vatni eykst massinn sem er fyrir ofan mann, þ.e.a.s. það
áhvílir meiri massi eftir því dýpra er farið. Þá er hægt að meta kraftinn
með
\[
	\uforce_\uacceleg = \rho \uvolume \uacceleg
		= \rho \uarea \ulengthh \uacceleg
\]
með innsetningu í þrýsting
\[
	\upressure = \frac{\rho \uarea \ulengthh \uacceleg}{\uarea}
		= \rho \ulengthh \uacceleg
\]
einfaldað gefur
\begin{equation}
	\upressure
		= \rho \ulengthh \uacceleg
\end{equation}
það sem er gott að muna er að þessi þrýstingur kemur frá öllum áttum, þá er
heildarkrafturinn núll. Það er einmitt vegna lögmáls Pascals, að vökvi dreifist
jafnt undir sama þrýstingi. Almennt getum við sagt að þrýstingurinn sem við upplifum
er líka þrýstingurinn frá andrúmsloftinu ásamt þrýstingnum frá vökvanum. Sem þýðir
\[
	\upressure_\text{heild} = \upressure + \upressure_0
\]
þar sem $\upressure_0$ er þrýstingur við yfirborð jarðar.

\subsection{Lögmál Arkimedesar}
Það er hægt að leiða út lögmálið út frá grunnforsendum, lögmálið er gefið sem
\begin{equation}
	\uforce_\text{upp} = \rho \uvolume \uacceleg
\end{equation}
þar sem $\uvolume$ er rúmmálið af hlutnum sem er dýft ofan í vökva og $\rho$ er
eðlismassi vökvans. Þá er hægt að leiða lögmálið út frá því að byrja skoða kraftana
sem verka á sívalning í vökva, heildarkrafturinn er
\begin{align*}
	\uforce_\text{heild} 
		&= \uforce_\text{botn} - \uforce_\text{top} \\
		&= \upressure_\text{botn} \uarea - \upressure_\text{top} \uarea \\
		&= \rho \ulengthh_\text{botn} \uacceleg \uarea - \rho \ulengthh_\text{top} \uacceleg \uarea \\
		&= \rho \uacceleg \uarea \left( \ulengthh_\text{botn} - \ulengthh_\text{top} \right)\\
		&= \rho \uacceleg \uarea \Delta \ulengthh \\
		&= \rho \uvolume \uacceleg
\end{align*}
fyrir sívalning gildir að $\uarea \Delta \ulengthh = \uvolume$. Sem gefur þegar
lögmál Arkimedes er jafngildi þessa, þ.e.a.s. 
$\uforce_\text{heild} = \uforce_\text{upp}$.


\begin{formalexample}
Bolti með rúmmálið $\SI{0.03}{\m\cubed}$ er haldið neðansjávar svo að hann er
kyrr, boltinn hefur massan $\SI{0.100}{\kg}$. Hversu stóran kraft þarf til þess
halda boltanum niðri?
\\[4 ex]
Kraftastefnan er valin til að vera jákvæð þegar krafturinn stefnir niður.
Heildarkrafturinn sem verkar á boltann er
\begin{align*}
	\uforce_\text{heild} &= \uforce_\text{upp} -\uforce_\text{g} 
		- \uforce_\text{ýta}
\end{align*}
þar sem boltinn er í kyrrstöðu þá er heildarkrafturinn núll
\begin{align*}
	\SI{0}{\N} &= \uforce_\text{upp} -\uforce_\text{g} 
		- \uforce_\text{ýta} \\
	\uforce_\text{ýta} &= \uforce_\text{upp} - \uforce_\text{g}
\end{align*}
krafturinn sem verkar vegna uppdrifs er
\begin{align*}
	\uforce_\text{upp} &= \rho \uvolume \uacceleg \\
		&= \SI{1000}{\kg\per\m\cubed} \times \SI{0.03}{\m\cubed}
			\times \SI{9.8}{\m\per\s\squared} \\
		&= \SI{294}{\N}
\end{align*}
og þyngdarkrafturinn
\begin{align*}
	\uforce_\text{g} &= \umass \uacceleg \\
		&= \SI{0.100}{\kg} \times \SI{9.8}{\m\per\s\squared} \\
		&= \SI{0.98}{\N}
\end{align*}
þá myndi ýtikrafturinn vera
\begin{align*}
	\uforce_\text{ýta} &= \uforce_\text{upp} - \uforce_\text{g} \\
		&= \SI{294}{\N} - \SI{0.98}{\N} \\
		&= \SI{293.02}{\N}
\end{align*}

\end{formalexample}
