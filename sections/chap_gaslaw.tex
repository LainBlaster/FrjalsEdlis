\chapter{Gasjafnan}\label{section:thermodynamics:idealgas}
Gas getur fundist í mörgum ástöndum, þá er gas sem er hagar sér eins og það er
samsett af kúlum sem hafa næga hreyfiorku til að yfirvinna bindisorkuna
sem gasatómin geta haft sín á milli. Þá kallast slíkt gas, \emph{kjörgas}
þar sem það er bara stjórnað af hitastigi, þrýstingi og rúmmáli. Þá skiptir
ekki máli hvort að um er að ræða atóm eða sameindir, svo lengi sem eigin
hreyfiorka markvert stærri en bindiorka á milli gaseindana. Þá er gasjafnan
gefin við
\begin{align}
	\upressure \uvolume &= \unumberN \uboltzmann \utemperature
\end{align}
líka er til önnur útgáfa sem mikið notuð í efnafræði
\begin{align}
	\upressure \uvolume &= \unumbern \ugasconstant \utemperature
\end{align}
Bæði tilvikin eru gaslögmálið fyrir kjörgas, þess ber að merkja að
þrýstingur margfaldaður með rúmmáli hefur sömu einingu og orka.
Reyndar er þetta orkan sem kjörgas hefur í samræmi við hitastigið sem gasið
finnst við. Fyrri jafnan skoðar orkuna á hverri eind sem finnst í gasinu á meðan
seinni skoðar magnið af orku sem finnst í einu móli af kjörgasi.
Þá er fasti Boltzmanns $\uboltzmann = \ugasconstantjk$ og gasfastinn
$\ugasconstant = \ugasconstantjkm$. 

\begin{formalexample}
	Ílát sem er með rúmmálið $0,1 \, \si{\metre\cubed}$ inniheldur 
	$\num{8,5} \, \si{\mole}$ af
	kjörgasi, þrýstingurinn í ílátinu er 
	$\num{2e5} \, \si{\pascal}$. Hvert er
	hitastigið í flöskunni?
	\\[4 ex]
	Hægt er að nota gasjöfuna með gasfastanum til að einangra hitan úr
	kjörgasjöfnunni
	\begin{align*}
		\num{2e5} \, \si{\pascal} \times 0,1 \, \si{\metre\cubed}
			&= \num{8,5} \, \si{\mole} \times 
				\ugasconstantjkm \times \utemperature 
			&& \Leftrightarrow \\
		\utemperature  &= \frac{ \num{2e5} \, \si{\pascal} \times 0,1 \, 
				\si{\metre\cubed} }
			{ \num{8,5} \, \si{\mole} \times \ugasconstantjkm } \\
		&= 283 \si{\kelvin} \\
		&= 10 \si{\celsius}
	\end{align*}
	sem þýðir að hitastigið er að meðaltali í ílátinu $10 \si{\celsius}$
\end{formalexample}
Þegar Boltzmanns fastinn er
notaður er talað um meðalorku per atóm í kjörgasi. 
Hægt er að lesa fastan $\uboltzmann = \ugasconstantjk$ sem 
$1,38 \si{\joule}$ af orku á hvert einasta atóm í gasinu fyrir hvert Kelvin.
\begin{formalexample}
	Loftþéttur stálkassi inniheldur $\num{5e24}$ atóm, þrýstingurinn í kassanum
	er mældur til að vera $\num{3e5} \, \si{\pascal}$ og hitastigið er mælt til
	að vera $\num{320} \, \si{\kelvin}$. Áætlaðu að gasið í kassanum sé
	kjörgas, finndu rúmmál kassans.
	\\[4 ex]
	Hægt er að nota gasjöfuna til að finna rúmmálið á kassanum með
	\begin{align*}
		\num{3e5} \, \si{\pascal} \times \uvolume
			&= \num{5e24} \times \ugasconstantjk \times \num{320} \, \si{\kelvin}
			&& \Leftrightarrow \\
		\uvolume
			&= \frac{ 
				\num{5e24} \times \ugasconstantjk \times 
				\num{320} \, \si{\kelvin}
				}{
				\num{3e5} \, \si{\pascal}
				}
			\\
			&= \frac{ 
				\num{2.2e4} \, \si{\joule}
				}{
				\num{3e5} \, \si{\pascal}
				}
			\\
			&= \num{0.073} \, \si{\metre\cubed} \\
			&= \num{73} \, \si{\deci\metre\cubed} \\
			&= \num{73} \, \si{\liter}
	\end{align*}
	Sem þýðir að kassinn fyllir $\num{73} \, \si{\liter}$ af rúmmáli. Hins vegar
	er hægt að finna út afhverju $\si{\joule\per\pascal} = \si{\metre\cubed}$,
	þegar við skoðum eininguna $\si{\joule} = \si{\newton\metre}$ og eininguna
	$\si{\pascal} = \si{\newton\per\metre\squared}$ sem gefur
	\begin{align*}
		\frac{\si{\joule}}{\si{\pascal}} 
			&= 
			\frac{\si{\cancel\newton\metre}}{
				\si{\cancel\newton\per\metre\squared}}
			= 
			\frac{\si{\metre}}{
				\si{\per\metre\squared}}
			= 
			\si{\metre\cubed}
	\end{align*}
\end{formalexample}
Það gildir líka fyrir gaslögmálið að orkan sem gasið hefur dreifist jafn í gegnum
lokað kerfi. Ef rúmmál, hiti eða þrýstingur breytist í lokuðu kerfi er orkan
óbreytt, sem þýðir að hægt er að setja
\begin{align}
	\frac{\upressure_1 \uvolume_1}{\utemperature_1}
		= \unumberN \uboltzmann
		= \unumbern \ugasconstant
		= \frac{\upressure_2 \uvolume_2}{\utemperature_2}
\end{align}
Ef tvö (eða fleiri) kerfi eru tengd saman þá hækkar orkan ekki en hins vegar 
breytist þrýstingur, rúmmál eða hitastig þegar gasið jafnast út. Eitthvað
af stærðunum sem stýra gildunum verða breytast annars hverfur orkan úr
kerfinu!
