\chapter{Kraftar}
% Intro to chapter 
Fyrst skilgreinum við hvað tregða er, það er mótþrói hlutar til að breyta
hraða sínum. Hlutur á hreyfingu hefur innbyggða tregðu sem lætur hlut halda
áfram í þá stefnu og hraða sem þegar hefur. Það sem getur breytt stefnu og hraða
hlutar eru kraftar og síðan eru nokkur lögmál sem voru framsett af Newton
sem lýsa þeim eiginleikum sem láta hlutir hafa áhrif á hvorn annan.
\begin{formaltext}
	\begin{description}
		\item[1. Tregða] Hlutur sem er á hreyfingu mun halda áfram í þá stefnu sem
			hann hefur, ásamt því að halda hraða sínum óbreyttum, nema kraftur
			breyti stefnu og/eða hraða hanns. Stærð tregðu, þ.e.a.s. mótþrói við
			breytingu í hreyfingu er skriðþungi, sem er margfeldi massa og hraða.
		\item[2. Kraftur] Kraftur er margfeldi, massa og hröðunar, enginn kraftur
			verkar ef hlutur er á jöfnum hraða og heldur sömu stefnu.
		\item[3. Gagnkraftur] Fyrir hvern kraft sem verkar á hlut, er jafnstór
			og gagnstæður kraftur sem verkar samtímis.
	\end{description}
\end{formaltext}
og til að byrja með er lögð áhersla á krafta fremur en tregðu og skriðþunga.
Frá öðru lögmáli Newtons er hægt að tala um mismunandi gerðir krafta sem allir
gegna hlutverki þegar reynt er að lýsa hegðun hluta, t.d. eru nokkrir
\begin{itemize}
	\item togkraftur
	\item þyngdarkraftur
	\item heildarkraftur
	\item núningskraftur
	\item rafkraftur
	\item segulkraftur
\end{itemize}
og þetta eru ekki allar mögulegar týpur af kröftum sem finnast. Allir kraftar
hafa eininguna $\unewton$ sem er SI-eining. Það er líka hægt að að skrifa kraft
sem $\ukilo\ugramm \uaccelems$.