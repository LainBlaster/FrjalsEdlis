%
% The definition of thermodynamic systems
%
\chapter{Varmalögmál}
Þegar varmi færist á milli staða er hægt að tala um lokuð og opin
kerfi, þeas. hvort að orka geti horfið úr kerfinu (opið kerfi).
Útfrá því eru nokkur lögmál kynnt á þeim grundvelli. Dagleg
reynsla segir okkur að hlutir reyna að nálgast hitastigið í
því umhverfinu. Þegar ísmoli er tekinn út í hitann þá bráðnar
hann þar sem ísmolinn móttekur varma frá umhverfinu. Þá er
er talað um jafnvægi sem næst á milli ísmolans og umhverfis.

\section{Núllta lögmál varmafræðinnar}
Í grófum dráttum er það:
\begin{quote}
	Þegar tveir hlutir (A og B) eru í jafnvægi við
	þriðja hlut (C), þá eru hlutir A og B í jafnvægi.
	Jafnvægi er skilgreint sem sama hitstig eða 
	jafnorkuflæði á milli hluta
\end{quote}
Lögmálið beinir athyglinni að því að hlutir með varma munu
ekki leita til þess að færa varma sinn nema kerfið sé í ójafnvægi.

\section{Fyrsta lögmál varmafræðinnar}
Orka í lokuðu kerfi er varðveitt, þá hverfur ekki orka í varmafærslu en í staðinn
fer hún í vinnu af hendi kerfisins. Hins vegar þegar orka bætist inní lokað
kerfi verður samanlögð innri orka kerfisins aukin og einhver hluti af orkunni
fer í annað en að auka innri orkuna (þeas. vinna).
\begin{quote}
	Samanlögð breyting á innri okru kerfisins er summan af viðbættri orku í 
	kerfið og vinnunni sem fer í að bæta orkunni við. Í jöfnuformi fæst
	\begin{align}
		\Delta \upotentialu &= \Delta \uthermaleng + \Delta \uworkw
	\end{align}
\end{quote}
Fyrir lokuð kerfi með engri breytingu er $\Delta \upotentialu = \SI{0}{\J}$. 
\begin{formalexample}
	Kjörgas móttekur $\SI{2.5}{\kJ}$ af orku og vinnan framkvæmd af gasinu
	er $\SI{300}{\J}$, hvað er breytingin í innri orku gasins?
	\\[4 ex]
	Þar sem vinnan er framkvæmd af gasinu þá er það orka sem gasið ætlur frá
	sér. Á sama tíma móttekur gasið orku, það gefur 
	\begin{align*}
		\Delta \upotentialu &= \Delta \uthermaleng + \Delta \uworkw \\
			&= \SI{2500}{\J} - \SI{300}{\J} \\
			&= \SI{2200}{\J}
	\end{align*}
	hækka innri orka kjörgasins um $\SI{2.2}{\kJ}$.
\end{formalexample}

\section{Annað lögmál varmafræðinnar}
Sem afleiðing af núllta og fyrsta lögmálinu, þá verður orka að flæða á 
milli hluta sem ná að komast í jafnvægi. Hins vegar segir annað lögmálið að 
varminn flæðir frá heitari hlutnum til kaldari hlutarins, oft er notuð óreiða 
til að lýsa þessu lögmáli. Lítil óreiða er kostnaðarsamari en mikil óreiða 
(við sama hitastig), það virkar andstætt almennri skynsemi en lítil óreiða 
samsvarar mikilli röðun og sem þarf að viðhalda. Lögmálið er oft skrifað sem
\begin{quote}
	Varmi leitar frá heitari hlut til kaldari hluts.
\end{quote}

\section{Þriðja lögmál varmfræðinnar}
Þegar efni kólnar minnkar óreiða efnis, ef efni er kælt til $0 \, \ukelvin$ þá
mun óreiða nálgast núll. Sem lögmál þá er þetta viðbót við annað lögmálið
til að gera fulla grein fyrir óreiðu.
\begin{quote}
	Efni við núll kelvin hefur núll óreiðu.
\end{quote}