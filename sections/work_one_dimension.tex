% work, energy definition
\section{Vinna}
Þegar kraftur verkar yfir vegalengd, þá kostar það orku eða leysir orku úr
læðingi. Almennt er vinna til gagns ef hún er jákvæð og ógagn
ef hún er neikvæð. Skilgreining á vinnu er:
\begin{equation}
	\uworkw = \uforce \Delta\ulengths
\end{equation}
þar sem $\uforce$ er krafturinn sem verkar og $\Delta\ulengths$ er vegalengdin
sem krafturinn verkar yfir. Þetta gildir um alla krafta, líka heildarkraft, sem
verður heildarvinna. Þá er sem dæmi hægt skoða þá krafta sem verka jafnt
yfir sömu vegalengd, sem dæmi er hægt að:
\begin{align*}
	\uforce_\text{heild} &= \uforce_\text{tog} - \uforce_\text{nún} \\
	\uforce_\text{heild}\Delta\ulengths &= 
		\uforce_\text{tog}\Delta\ulengths - \uforce_\text{nún}\Delta\ulengths \\
	\uworkw_\text{heild} &= \uworkw_\text{tog} - \uworkw_\text{nún}
\end{align*}
svo lengi sem kraftarnir verka yfir sömu vegalengd. Þetta svipar til að
heildarkrafts, heildarorka er samansafn af öllum orkum sem vinna til gagns eða
ógagns.
%
\begin{formalexample}
Kassi er dreginn $\SI{5}{\m}$ yfir flöt, núningskrafturinn sem verkar á hreyfistefnunni er
$\SI{10}{\N}$ og er togaður áfram með $\SI{25}{\N}$ krafti.
Hver er heildarvinnan sem er framkvæmd?
\\[4 ex]
Vinnan sem er framkvæmd af togkraftinum er
\[
	\uworkw_\text{tog} = \uforce_\text{tog} \Delta\ulengths 
		= \SI{25}{\N} \times \SI{5}{\m}
		= \SI{125}{\J}
\]
og vinnan sem er framkvæmd af núningskraftinum er
\[
	\uworkw_\text{nún} = \uforce_\text{nún} \Delta\ulengths 
		= \SI{10}{\N} \times \SI{5}{\m}
		= \SI{50}{\J}
\]
þá er heildarvinnan
\[
	\uworkw_\text{heild} = \uworkw_\text{tog} - \uworkw_\text{nún}
		= \SI{125}{\J} - \SI{50}{\J}
		= \SI{75}{\J}
\]
þó það kostaði $\SI{125}{\J}$ af orku að toga kassann áfram, þá var ekki
nema $\SI{75}{\J}$ sem gátu nýst í að koma kassanum áfram, afgangurinn fór í
núning sem verkaði á móti hreyfingunni.
\end{formalexample}

\section{Skriðorka}
Út frá skilgreiningu um vinnu er hægt að finna út samhengi á milli orku hlutar
og hraða hlutar. Hlutur byrjar í kyrrstöðu ($\uspeed_0 = \SI{0}{\m\per\s}$) 
og nær hraðanum $\uspeed$
með jafnri hröðun og skoðum vinnuna sem fer í að koma hlutnum áfram með
krafti $\uforce$:
\begin{equation*}
	\uworkw = \uforce_\text{heild} \Delta\ulengths
		= \umass \uaccelea \Delta\ulengths
		= \umass \left( \frac{\uspeed^2 - \uspeed^2_0}{2} \right)
		= \frac{1}{2} \umass \uspeed^2
		\equiv \ukinetick
\end{equation*}
sem er orkan sem hluturinn hefur náð við að hraða sér
upp í hraðan $\uspeed$. Þetta er almennt kallað skriðorka
\begin{equation}
	K = \frac{1}{2} \umass \uspeed^2
\end{equation}
að endurtaka sömu útleiðslu nema í stað þess að byrja í kyrrstöðu þá er byrjað
á hraðanum $\uspeed_0 \neq 0$
\begin{equation*}
	\uworkw
		= \umass \uaccelea \Delta\ulengths
		= \umass \left( \frac{\uspeed^2 - \uspeed^2_0}{2} \right)
		= \frac{1}{2} \umass \uspeed^2 - \frac{1}{2} \umass \uspeed^2_0
		= \ukinetick - \ukinetick_0
		= \Delta\ukinetick
\end{equation*}
þá gefur breytingin í hraða vinnuna sem hefur verið framkvæmd á meðan
hraðabreytingunni stóð. Þetta er kallað vinnulögmálið
\begin{equation}
	\uworkw = \Delta\ukinetick
\end{equation}

\section{Stöðuorka}
Þegar hlutur ferðast í þyngdarsviði verkar kraftur á hann, það er ávallt
þyngdarkraftur sem verkar í átt að miðju jarðar. Vinnan sem er framkvæmd að 
\emph{minnsta kosti} til að færa hlut upp hæðarmismun á jöfnum hraða er samanlagt
núll, þ.e.as. $\uforce_\text{heild} = \SI{0}{\N}$ sem þýðir $\uforce_\text{tog} = 
\uforce_\text{g}$, sem gefur
\begin{align*}
	\uworkw_\text{heild} &= \uworkw_\text{tog} - \uworkw_\text{g} = \SI{0}{\J} \\
	\uworkw_\text{tog} &= \uworkw_\text{g} = \umass \uacceleg \Delta\ulengthh
\end{align*}
sú vinna sem var framkvæmd við að færa hlutinn upp hæðarmismuninn $\Delta\ulengthh$. 
Ef við sleppum hlutnum, þá fellur hann í frjálsu falli og við fáum vinnuna sem var
framkvæmd leyst úr læðingi. Vinnan geymist í þyngdarsviðinu sem orka, og kallast
\emph{geyminn orka} þar sem vinnan framkvæmd einungis til að breyta hæð hlutarins
er endurheimtanleg. Dæmi um ógeymda orku, er núningur, vinnan sem er framkvæmd og
fer í núning er ekki hægt að endurheimta á sama máta og úr þyngdarsviði. Sem
gefur að orkan sem geymist í einsleitu þyngdarsviði jarðar hefur stærðina
\begin{equation}
	\Delta \upotentialu = \umass \uacceleg \Delta\ulengthh
\end{equation}
varðandi hæðarmismuninn $\Delta\ulengthh = \ulengthh_2 - \ulengthh_1$, þá
þarf að velja \emph{núllpunkt} ($\ulengthh_1$) sem er upphafsstaðsetning. Þá 
getur lokastaðsetning ($\ulengthh_2$) verið fyrir neðan eða ofan upphafstaðsetningu.

\begin{formalexample}
Lóð með massann $\SI{5}{\kg}$ er togað upp $\SI{10}{\m}$ hæð frá jörðu upp á húsþak.
Hver er stöðuorka lóðsins séð frá jörðinni? En séð frá húsþakinu? Síðan er lóðinu
sleppt, hver er stöðuorka lóðsins þá?
\\[4 ex]
Séð frá jörðu, þá er stöðuorkan
\begin{align*}
	\Delta \upotentialu &= \umass \uacceleg \Delta\ulengthh \\
		&= \SI{5}{\kg} \times \SI{9.8}{\m\per\s\squared} 
			\times ( \SI{10}{\m} - \SI{0}{\m} ) \\
		&= \SI{490}{\J}
\end{align*}
Séð frá húsþakinu, þá er stöðuorkan
\begin{align*}
	\Delta \upotentialu &= \umass \uacceleg \Delta\ulengthh \\
		&= \SI{5}{\kg} \times \SI{9.8}{\m\per\s\squared} 
			\times (\SI{0}{\m}) \\
		&= \SI{0}{\J}
\end{align*}
Hins vegar þegar lóðið dettur niður
\begin{align*}
	\Delta \upotentialu &= \umass \uacceleg \Delta\ulengthh \\
		&= \SI{5}{\kg} \times \SI{9.8}{\m\per\s\squared}
			\times (-\SI{10}{\m} - \SI{0}{\m} ) \\
		&= -\SI{490}{\J}
\end{align*}
Þetta þýðir í stuttu máli að lóðið eykur alltaf innri orku sína í formi stöðuorku
þegar lóðið fer gagnstætt þyngdarsviði og þegar lóðið fylgir stefnu þyngdarsviðsins
þá tapar lóðið innri orkunni sinni sem var geyminn í því.
\end{formalexample}

\section{Vélræn orka}
Hlutur á hreyfingu hefur bæði skriðorku og stöðuorku, samanlagt eru þessar stærðir
vélræn orka hlutars. Sem er 
\begin{equation}
	\umechenergy = \upotentialu + \ukinetick
\end{equation}
vélræn orka varðveitist ef það eru \emph{engir} núningskraftar eða aðrir kraftar en þeir
sem eru gefnir af þyngdarsviði. Sem gefur mjög hentuga leið til að vita hver
hraði hlutar ætti að vera ef orkan er varðveitt, samtímis er hægt að fá
upplýsingar um tapið af orku sem fer í núning.
%
\begin{formalexample}
Kúla er í $\SI{15}{\m}$ hæð yfir jörðu, kúlan er $\SI{10}{\kg}$. Hver er 
stöðuorka kúlunar? Ef kúlan er látin falla,
hversu mikil verður skriðorka og hraði kúlunnar við lendingu?
\\[4 ex]
Hæðarmismunurinn við jörðu er $\Delta\ulengthh = \SI{15}{\m}$, stöðuorkan er gefin til
að vera
\[
	\Delta\upotentialu = \umass \uacceleg \Delta \ulengthh
		= \SI{10}{\kg} \times \SI{9.8}{\m\per\s\squared}
			\times \SI{15}{\m}
		= \SI{1470}{\J}
\]
þar sem enginn núningur er á meðan kúlan fellur (áætlum hverfandi loftmótstaða) þá er
ekkert tap á vélrænni orku sem þýðir að öll stöðuorkan breytist í skriðorku. Sem gefur
að $\Delta\umechenergy = 0$, þeas. engin breyting sem gefur að
\[
	\Delta\upotentialu = \ukinetick = \SI{1470}{\J}
\]
sem er hægt að umskrifa til
\[
	\ukinetick = \frac{1}{2} \umass \uspeed^2 
		\Leftrightarrow
		\uspeed = \sqrt{\frac{2 \ukinetick}{\umass}} 
			= \sqrt{\frac{\num{2} \times \SI{1470}{\J}
				}{\SI{10}{\kg}}}
			= \SI{17.1}{\m\per\s}
\]
sem er hraði kúlunnar rétt áður fyrir lendingu.
\end{formalexample}
%
\begin{formalexample}
Bíll hefur upphafshraðan $\SI{5}{\m\per\s}$ við toppinn á $\SI{40}{\m}$ háa brekku,
bíllinn rennur vélvana (s.s. á þess að mótorinn hjálpar) niður brekkuna og hefur
hraðan $\SI{20}{\m\per\s}$ við botninn á brekkunni. 
Massi bílsins er $\SI{1000}{\kg}$
og samanlagt rennur bíllinn vegalengdina $\SI{70}{\m}$.
Hver er breytingin í vélrænni orku bílsins? Hversu mikið af orkunni fer í annað
en hreyfiorku eða stöðuorku? Hver er meðalkrafturinn sem verkar á móti hreyfingu
bílsins?
\\[4 ex]
Hæðarmismunurinn við jörðu er $\Delta\ulengthh = \SI{40}{\m}$, og upphafshraðinn
er $\SI{5}{\m\per\s}$ áður en bíllinn rennur niður brekkuna sem þýðir að vélræn
orka bílsins áður en hann rennur niður er
\begin{align*}
	\umechenergy_\text{fyrir} 
		&= \umass \uacceleg \Delta\ulengthh + \frac{1}{2} \umass \left( \uspeed_1 \right)^2 \\
		&= \SI{1000}{\kg} \times \SI{9.8}{\m\per\s\squared} \times \SI{40}{\m} 
			+ \frac{1}{2} \times \SI{1000}{\kg} 
				\times \left( \SI{5}{\m\per\s} \right)^2 \\
		&= \SI{405}{\kJ}
\end{align*}
við botninn á brekkunni er vélræn orka bílsins
\begin{align*}
	\umechenergy_\text{eftir} 
		&= \umass \uacceleg \Delta\ulengthh + \frac{1}{2} \umass \left( \uspeed_2 \right)^2 \\
		&= \SI{1000}{\kg} \times \SI{9.8}{\m\per\s\squared} \times \SI{0}{\m} 
			+ \frac{1}{2} \times \SI{1000}{\kg}
				\times \left( \SI{20}{\m\per\s} \right)^2 \\
		&= \SI{200}{\kJ}
\end{align*}
þá er breytingin í vélrænni orku
\[
	\Delta\umechenergy = \umechenergy_\text{eftir} - \umechenergy_\text{fyrir} 
		= \SI{200}{\kJ} - \SI{405}{\kJ} = -\SI{205}{\kJ}
\]
sem er tapið af vélrænni orku, orkan hefur ekki varðveist á leiðinni niður og
megnið (hátt í $50 \%$) hefur farið að reyna yfirvinna núning og mótstöðukrafta.
Ef vélræn orka varðveitist ekki þá er kraftur sem er einhverskonar mótstöðukraftur
við hreyfingu bílsins. Magnið af orku sem hefur verið gefið úr vélrænni orku er
mikið og hefur farið í vinnu af mótstöðukröftum
\[
	\uworkw = - \uforce_\text{mót} \Delta\ulengths
		\Leftrightarrow
		\uforce_\text{mót} = - \frac{\uworkw}{\Delta\ulengths}
			= - \frac{ -\SI{205}{\kJ} }{ \SI{70}{\m} }
			= - \SI{2.93}{\kN}
\]
\end{formalexample}
%
\begin{formalexample}
Mótor togar $\SI{250}{\kg}$ lyftu upp með 
kraftinum $\SI{2.5}{\kN}$, skinnurnar sem lyftan fer
eftir mynda núningskraft uppá $\SI{250}{\N}$. Lyftan 
ferðast upp hæðina $\SI{25}{\m}$,
hversu mikla orku þarf mótorinn að láta frá sér til að lyftan kemst upp?
\\[4 ex]
Orkan sem lyfta þarf að fá í form stöðuorku er
\[
	\upotentialu = \umass \uacceleg \ulengthh 
		= \SI{250}{\kg} \times \SI{9.8}{\m\per\s\squared}
			\times \SI{25}{\m}
		= \SI{61250}{\J}
\]
vinnan sem núningur framkvæmir gagnsætt mótori er
\[
	\uworkw = \SI{250}{\N} \times \SI{25}{\m}
		= \SI{6250}{\J}
\]
þá er heildarvinna jöfn orkunni sem fór í að toga lyftuna upp
\begin{align*}
	\uworkw_\text{heild} &= \uworkw_\text{tog} - \uworkw_\text{nún}\\
	\uworkw_\text{tog} &= \uworkw_\text{heild} + \uworkw_\text{nún}\\
		&= \SI{61250}{\J} + \SI{6250}{\J} \\
		&= \SI{67500}{\J}
\end{align*}
Þá þarf mótorinn að gefa minnst þetta magn af orku til þess að lyftan komist upp
$\SI{25}{\m}$.
\end{formalexample}
%
