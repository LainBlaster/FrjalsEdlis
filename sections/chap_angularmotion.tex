% Chap angular motion
\chapter{Hringhreyfing}
Þegar hlutir hreyfast í hringhreyfingu þarf að taka mið af því hvernig hnitakerfið er
uppsett. Í hefðbundnu rétthyrndu hnitakerfi þarf að lýsa hreyfingunni með vigrum
og í tvívídd. Hringhreyfing er líka eitt af þeim svæðum þar sem skiptir máli
hvar upphafsstaðsetning hnitakerfsins er. Þetta þýðir að við getum upplifað
sömu aðstæður sem kraftar verka á hluti á nokkra mismunandi og oft furðulega
máta. Sem hluti af því þarf að skilgreina tregðukerfi og heildarkraftinn
uppá nýtt.

\section{Tregðukerfi}
Þegar við erum í lyftu fáum við upplifun að það er kraftur sem togar okkur
niður, á þvert við tilfinninguna okkar þá er reyndar enginn kraftur sem togar
okkur niður. Heldur er lyftan að ýta okkur upp og við erum að upplifa kraftinn
sem kemur við að breyta hraða okkar. Séð utan frá, þá er lyftan að ýta okkur
upp, síðan er gagn kraftur á milli okkar og lyftunar, þá er heildarkrafturinn
jákvæður og stefnir upp. S.s. skrifað í kröftum
\begin{align*}
	\uforce_\text{heild} 
		&= \uforce_\text{tog, lyfta} - \uforce_\text{g, lyfta} - \uforce_\text{g, persóna}\\
	(\umass_\text{lyfta} + \umass_\text{persóna}) \uaccelea
		&= \uforce_\text{tog, lyfta} - \uforce_\text{g, lyfta} - \uforce_\text{g, persóna}\\
\end{align*}
þar sem $\uforce_\text{heild, lyfta}$ er krafturinn sem lyftan getur lyft sér með
án þess að hafa farþega. Ef við skoðum kraftinn sem verkar á okkur í lyftunni
\begin{align*}
	\uforce_\text{heild, persóna}
		&= \uforce_\text{þver} - \uforce_\text{g} \\
	\umass_\text{persóna} \uaccelea
		&= \uforce_\text{þver} - \uforce_\text{g} \\
	\uforce_\text{þver}
		&= \umass_\text{persóna} \uaccelea + \uforce_\text{g} \\
\end{align*}
þverkrafturinn $\uforce_\text{þver}$ er stærri þyngdarkrafturinn, og þverkraftur
er gagnkraftur sem verkar gagnstætt á lyftubotninn. Þar sem persónan helst á
gólfi lyftunar, þá er gagnkrafturinn sem verkar á lyftuna jafn stór og gagnstæður
þverkrafts persónunnar. Þá er heildkrafturinn fyrir lyftuna staka
\begin{align*}
	\uforce_\text{heild, lyfta}
		&= \uforce_\text{tog, lyfta} - \uforce_\text{g, lyfta} - \uforce_\text{þver} \\
	\umass_\text{lyfta} \uaccelea
		&= \uforce_\text{tog, lyfta} - \uforce_\text{g, lyfta} - (\umass_\text{persóna} \uaccelea + \uforce_\text{g} ) \\
	\umass_\text{lyfta} \uaccelea
		&= \uforce_\text{tog, lyfta} - \uforce_\text{g, lyfta} - \umass_\text{persóna} \uaccelea - \uforce_\text{g} \\
	\umass_\text{lyfta} \uaccelea + \umass_\text{persóna} \uaccelea
		&= \uforce_\text{tog, lyfta} - \uforce_\text{g, lyfta}  - \uforce_\text{g} \\
	(\umass_\text{lyfta} + \umass_\text{persóna}) \uaccelea
		&= \uforce_\text{tog, lyfta} - \uforce_\text{g, lyfta}  - \uforce_\text{g} \\
	\uforce_\text{heild, allt}
		&= \uforce_\text{tog, lyfta} - \uforce_\text{g, lyfta}  - \uforce_\text{g} \\
\end{align*}
sem sagt, þó að við blöndum saman tveim ,,kerfum'', þá er ennþá sama niðurstaða
eins og það væri fyrir utan lyftuna. 

Þegar við erum í hringhreyfingu þá er heildarkrafturinn sem verkar á hlut með
stefnuna í átt að miðju hringsins. Annars væri hluturinn að fara beint meðfram
þeirri stefnu sem hraði hlutarins hefur á hringferlinum. S.s. þá er heildarkrafturinn
sem verkar á hlut er
\begin{equation}
	\bar\uforce_\text{heild, hring} = \bar\uforce_\text{mið}
\end{equation}
þá er líka mikilvægt að gera sér grein fyrir því að heildarkrafturinn
er líka
\[
	\bar\uforce_\text{heild, hring} 
		= \bar\uforce_\text{mið}
		= \umass \bar\uaccelea_\text{mið}
		= \sum \text{ summa krafta á hlut }
\]
sem leiðir af sér að hröðunin sem hluturinn hefur verður að stefna í átt að
miðju hringferlsins. Þá þarf einmitt að finna út hvernig vigurstærðin 
$\bar \uaccelea$ lítur út, sem er kölluð miðsóknarhröðun.

\section{Miðsóknarhröðun}
Í þessum hluta er markmiðið að sýna stærð miðsóknarhröðunnar er gefin til að vera
\begin{equation}
	\uaccelea_\text{mið} = \frac{\uspeed}{\ulengthr}
\end{equation}
fyrst til að komast að þessari niðurstöðu þarf að skilgreina hringferil
\begin{align*}
	\bar \ulengthr = 
		\left( 
			\begin{array}{c}
				\ulengthr \cos( \theta(\utime) )\\
				\ulengthr \sin( \theta(\utime) )\\
			\end{array}
		\right)
\end{align*}
þar sem radíus hringsins er $\ulengthr$ og hornið er $\theta(\utime)$
en hins vegar er hornið sem hluturinn ferðast meðfram hringferilunum er háð
tíma, nánar tiltekið
\[
	\theta = \omega \utime + \theta_0
\]
þar sem $\omega$ er hornhraði, þeas hversu margar gráður/radíanar hornið breytist
með á hverri sekúndu, upphafshornið $\theta_0$ er venjulega núll. 
Þá verður staðsetningin $\bar \ulengthr$ meðfram hringferilunum
\begin{align*}
	\bar \ulengthr = 
		\left( 
			\begin{array}{c}
				\ulengthr \cos( \omega \utime )\\
				\ulengthr \sin( \omega \utime)\\
			\end{array}
		\right)
\end{align*}
og hraði er $\bar \uspeed = \frac{d \bar \ulengthr}{ d \utime}$, sem gefur
\begin{align*}
	\bar \uspeed = 
		\left( 
			\begin{array}{c}
				- \ulengthr \omega \sin( \omega \utime )\\
				\ulengthr \omega \cos( \omega \utime)\\
			\end{array}
		\right)
\end{align*}
þá er stærð hraðavigursins
\begin{align*}
	\left\| \bar \uspeed \right\| &= 
		\left\| \left( 
			\begin{array}{c}
				- \ulengthr \omega \sin( \omega \utime )\\
				\ulengthr \omega \cos( \omega \utime)\\
			\end{array}
		\right)
		\right\| \\
	 &= 
		\sqrt{ \left( - \ulengthr \omega \sin( \omega \utime ) \right)^2 
			+ \left( \ulengthr \omega \cos( \omega \utime) \right)^2 }
		\\
		&= \ulengthr \omega 
			\sqrt{ \sin^2( \omega \utime ) +  \cos^2( \omega \utime ) } \\
		&= \ulengthr \omega 
			\sqrt{ 1 } \\
		&= \ulengthr \omega
\end{align*}
sem er hægt að skrifa sem
\[
	\uspeed = \left\| \bar \uspeed \right\| = \ulengthr \omega
\]
þetta er einungis stærð vigursins, stefnan er gefin við $\bar \uspeed$. Hröðun
er skilgreind sem breyting hraða á móti tíma, eða $\bar \uaccelea =
\frac{ d \bar \uspeed }{d \utime}$. Þá er hröðun sem vigurstærð
\begin{align*}
	\bar \uspeed = 
		\left( 
			\begin{array}{c}
				- \ulengthr \omega^2 \cos( \omega \utime )\\
				- \ulengthr \omega^2 \sin( \omega \utime)\\
			\end{array}
		\right)
\end{align*}
þá er stærð hraðavigursins 
\begin{align*}
	\left\| \bar \uaccelea_\text{mið} \right\| &= 
		\left\| \left( 
			\begin{array}{c}
				- \ulengthr \omega^2 \cos( \omega \utime )\\
				- \ulengthr \omega^2 \sin( \omega \utime)\\
			\end{array}
		\right)
		\right\| \\
	 &= 
		\sqrt{ \left( - \ulengthr \omega^2 \cos( \omega \utime ) ) \right)^2 
			+ \left( - \ulengthr \omega^2 \sin( \omega \utime) \right)^2 }
		\\
		&= \ulengthr \omega^2 
			\sqrt{ \sin^2( \omega \utime ) +  \cos^2( \omega \utime ) } \\
		&= \ulengthr \omega^2 
			\sqrt{ 1 } \\
		&= \ulengthr \omega^2
\end{align*}
sem er hægt að skrifa sem
\[
	\uaccelea_\text{mið} = \left\| \bar \uaccelea_\text{mið} \right\| = \ulengthr \omega^2
\]
og þar sem hornhraði er $\omega = \frac{\uspeed}{\ulengthr}$ þá er
\[
	\uaccelea_\text{mið} = \ulengthr \left( \frac{\uspeed}{\ulengthr} \right)^2
		= \frac{\uspeed^2}{\ulengthr}
\]
sem sýnir að stærð miðsóknarhröðunar er það sem var fyrst haldið fram. Stefna
miðsóknarhröðunnar er hinsvegar alltaf í átt að miðju hringsins, sem er reyndar
ástæðan fyrir nafninu \emph{miðsóknar}hröðun.

\subsection{Tíðni og hringhreyfing}
Hornhraði er stærð sem er skilgreind til að vera radíanar á sekúndu, en tíðni er
skilgreind til að vera snúningar á sekúndu. Þá er samhengið á milli tíðni
og hornahraða
\begin{equation}
	\omega = 2 \pi \ufrequency
\end{equation}
þá er hægt að umskrifa miðsóknarhröðun á marga máta
\begin{align*}
	\uaccelea_\text{mið} &= 
		\frac{\uspeed^2}{\ulengthr}\\
	 &= 
		\ulengthr \omega^2\\
	 &=
		4 \pi^2 \ufrequency^2 \ulengthr
\end{align*}
og umferðartíminn fyrir hlut í hringhreyfingu er
\[
	\uorbitaltime = \frac{2\pi}{\omega}
\]
umferðartími er tíminn sem það tekur hlutinn að fara \emph{einn hring}. T.d.
er umferðartími jarðar $\approx 1 \uyear$.

\section{Miðsóknarkraftur}
Þar sem heildarkrafturinn sem verkar á hlut þarf að vera jafn miðsóknarkrafti, 
þá er
\begin{equation}
	\uforce_\text{heild} = \uforce_\text{mið} = \umass \frac{\uspeed^2}{\ulengthr}
\end{equation}
sem er hægt að setja upp á mismunandi máta. Þá verður heildarkrafturinn séð frá
tregðukerfi sem er utan hringhreyfingar. Það verður vandamál að setja upp
alla möguleika sem geta lýst kröftunum á meðan hringhreyfingu stendur. Sýnidæmi
eru hentug til að lýsa heildakröftum og miðsóknarkrafti.
\begin{formalexample}
Kúla er í hringhreyfingu í láréttu plani, massi kúlunnar er $2 \ukilo\ugramm$ og
lengd taugarinnar $1,2 \umeter$.
Taugin (bandið) sem heldur kúlunni þolir mest $20 \unewton$ kraft áður en taugin
slitnar. Hver er hámarkshraði kúlunnar áður en taugin slitnar?
\\[4 ex]
Heildarkrafturinn sem verkar á hlutinn er
\[
	\uforce_\text{heild} 
		= \uforce_\text{mið} 
		= \umass \frac{\uspeed^2}{\ulengthr}
		= \uforce_\text{taug}
\]
sem er hægt leysa sem
\begin{align*}
	\umass \frac{\uspeed^2}{\ulengthr} &= 
		\uforce_\text{taug}\\
	2 \ukilo\ugramm \cdot \frac{\uspeed^2}{1,2 \umeter} &= 
		20 \unewton\\
	\uspeed^2 &= 
		\frac{1,2 \umeter \cdot 20 \unewton }{ 2 \ukilo\ugramm }\\
	\uspeed &= 
		\sqrt{ \frac{1,2 \umeter \cdot 20 \unewton }{ 2 \ukilo\ugramm } }\\
	 &= 
		3,46 \uspeedms
\end{align*}
sem er hámarkshraði áður en bandið slitnar.
\end{formalexample}
\begin{formalexample}
Kúla er í hringhreyfingu í lóðréttu plani, massi kúlunnar er $2 \ukilo\ugramm$,
lengd taugarinnar $1,2 \umeter$ og hraði kúlunnar er $5 \uspeedms$.
Hver er togkrafturinn í tauginn þegar kúlan er í toppnum eða botni hringhreyfinginnar?
\\[4 ex]
Heildarkraftuinn hefur núna líka þyngdarkraft sem verkar á taugina við toppinn
\begin{align*}
	\uforce_\text{heild} 
		= \uforce_\text{taug} + \uforce_\text{g}\\
	\umass \frac{\uspeed^2}{\ulengthr} &= 
		\uforce_\text{taug} + \umass \uacceleg\\
	2 \ukilo\ugramm \cdot \frac{\left( 5 \uspeedms \right)^2}{1,2 \umeter} &= 
		\uforce_\text{taug} + 2 \ukilo\ugramm \cdot 9,8 \uaccelems\\
	41,7 \unewton &= 
		\uforce_\text{taug} + 19,6 \unewton\\
	\uforce_\text{taug} &= 
		41,7 \unewton - 19,6 \unewton\\
	 &= 
		22,1 \unewton
\end{align*}
aftur á móti við botninn á hringferlinum þá er heildarkrafturinn
\begin{align*}
	\uforce_\text{heild} 
		= \uforce_\text{taug} - \uforce_\text{g}\\
	\umass \frac{\uspeed^2}{\ulengthr} &= 
		\uforce_\text{taug} - \umass \uacceleg\\
	2 \ukilo\ugramm \cdot \frac{\left( 5 \uspeedms \right)^2}{1,2 \umeter} &= 
		\uforce_\text{taug} - 2 \ukilo\ugramm \cdot 9,8 \uaccelems\\
	41,7 \unewton &= 
		\uforce_\text{taug} - 19,6 \unewton\\
	\uforce_\text{taug} &= 
		41,7 \unewton + 19,6 \unewton\\
	 &= 
		61,3 \unewton
\end{align*}
togkrafturinn verður þá að vera stærri við botninn en við toppinn á hringferlinum.
\end{formalexample}
\begin{formalexample}
Kúla er í hringhreyfingu í lóðréttu plani,
lengd taugarinnar $1,2 \umeter$.
Hver er minnsti hraði sem hægt er að sleppa með til að halda kúlunni í hringhreyfingu?
\\[4 ex]
Til að finna minnsta hraðan þá þarf að finna nákvæmlega þann punkt sem
miðsóknarkraftur er ekki undir áhrifum togkrafts taugarinnar og einungis
þyngdarkraftur hefur áhrif á hlutinn. S.s. $\uforce_\text{taug} = 0$ og
í toppstöðu getur þyngdarkrafturinn búið til kraft sem verkar á hlutinn og
sér til þess að hann heldur hringhreyfingu
\begin{align*}
	\uforce_\text{heild} 
		&= \uforce_\text{g}\\
	\umass \frac{\uspeed^2}{\ulengthr} &= 
		\umass \uacceleg\\
	\frac{\uspeed^2}{\ulengthr} &= 
		\uacceleg\\
	\frac{\uspeed^2}{1,2 \umeter} &= 
		9,8 \uaccelems\\
	\uspeed^2 &= 
		9,8 \uaccelems \cdot 1,2 \umeter \\
	\sqrt{ \uspeed^2 }  &= 
		\sqrt{ 9,8 \uaccelems \cdot 1,2 \umeter } \\
	\uspeed  &= 
		\sqrt{ 9,8 \uaccelems \cdot 1,2 \umeter } \\
	&=
		3,43 \uspeedms
\end{align*}
\end{formalexample}
\begin{formalexample}
Lóð hengur í bandi, massi lóðsins
er $5 \ukilo\ugramm$. Hver er krafturinn í bandinu þegar lóðið er
dregið upp í $90 \udeg$ horn upp og látið falla?
\\[4 ex]
Það er hægt að finna hraðan sem lóðið hefur þegar það er komið í lægstu stöðu eftir
að því er sleppt frá $90\udeg$ horninu. Í hæðstu stöðu er lóðið
í $\ulengthr$ hæð, síðan dettur lóðið og öll stöðuorka hlutarins er
færð í skriðorku
\begin{align*}
	\umass \uacceleg \ulengthr
		&= \frac{1}{2} \umass \uspeed^2\\
	\uacceleg \ulengthr 
		&= \frac{1}{2} \uspeed^2\\
	\uspeed^2 
		&= 2 \uacceleg \ulengthr
\end{align*}
heildkrafturinn á bandið og lóðið við botninn á hringhreyfingu lóðsins er
\begin{align*}
	\uforce_\text{heild}
		&= \uforce_\text{taug} - \uforce_\text{g}\\
	\umass \frac{\uspeed^2}{\ulengthr}
		&= \uforce_\text{taug} - \umass \uacceleg\\
	\umass \frac{2 \uacceleg \ulengthr}{\ulengthr}
		&= \uforce_\text{taug} - \umass \uacceleg\\
	2 \umass \uacceleg
		&= \uforce_\text{taug} - \umass \uacceleg\\
	\uforce_\text{taug}
		&= 2 \umass \uacceleg + \umass \uacceleg\\
		&= 3 \umass \uacceleg \\
		&= 3 \cdot 5 \ukilo\ugramm \cdot 9,8 \uaccelems \\
		&= 147 \unewton
\end{align*}
sem er krafturinn sem taugin upplifir.
\end{formalexample}
